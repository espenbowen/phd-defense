\documentclass[aspectratio=1610]{beamer}
\usetheme{bjeldbak}
\usepackage{xspace}
\usepackage{graphicx}
\usepackage{textpos}
\usepackage{subfigure}
\usepackage{pifont}
\usepackage[super]{nth}
\usepackage{relsize}
\usepackage{amsmath}
\usepackage{bm}

\setbeamertemplate{section page}{
  \begin{centering}
    \begin{beamercolorbox}[sep=12pt,center]{part title}
      \huge
      \insertsection
    \end{beamercolorbox}
  \end{centering}
}

\definecolor{anti-flashwhite}{rgb}{0.95, 0.95, 0.96}

\setbeamercolor{section in toc}{fg=anti-flashwhite}

\setbeamertemplate{section page}{
  \begin{centering}
    \begin{beamercolorbox}[sep=12pt,center]{part title}
      \huge
      \insertsection
    \end{beamercolorbox}
  \end{centering}
}

\usepackage{ifthen}
\usepackage{mciteplus} 
\newboolean{uprightparticles}
\setboolean{uprightparticles}{false} %Set true for upright particle symbols
\usepackage{xspace} 
\usepackage{upgreek}

\input{lhcb-symbols-def}
\input{spd-symbols-def}
\input{tracking-symbols-def}
\input{quote}

\def \Cseven {\ensuremath{\mathcal{C}_{7}^{(\ensuremath{\prime})}}\xspace}
\def \Cnine {\ensuremath{\mathcal{C}_{9}^{(\ensuremath{\prime})}}\xspace}
\def \Cten {\ensuremath{\mathcal{C}_{10}^{(\ensuremath{\prime})}}\xspace}
\def \Cnineten {\ensuremath{\mathcal{C}_{9,10}^{(\ensuremath{\prime})}}\xspace}

\usepackage{appendixnumberbeamer}
\expandafter\def\expandafter\insertshorttitle\expandafter{%
  \insertshorttitle\hfill\insertframenumber\,/\,\inserttotalframenumber}

\usepackage[linewidth=1pt]{mdframed}
\newenvironment{myenv}[1]
{\mdfsetup{
  frametitle={\colorbox{white}{\space#1\space}},
  innertopmargin=3pt,
  frametitleaboveskip=-\ht\strutbox,
  frametitlealignment=\center
}
\begin{mdframed}%
}
{\end{mdframed}}

\setbeamertemplate{blocks}[rounded][shadow=false]
\setbeamercolor{block body}{bg=barcolor!40,fg=black}
\setbeamercolor{block title}{bg=barcolor!20,fg=black}

\usepackage{tikz}
\usetikzlibrary{shapes,arrows}

\definecolor{babyblue}{rgb}{0.54, 0.81, 0.94}
\definecolor{byzantine}{rgb}{0.74, 0.2, 0.64}
\definecolor{anti-flashwhite}{rgb}{0.95, 0.95, 0.96}
\definecolor{burgundy}{rgb}{0.5, 0.0, 0.13}
\definecolor{burntorange}{rgb}{0.8, 0.33, 0.0}
\definecolor{cadmiumorange}{rgb}{0.93, 0.53, 0.18}
\definecolor{bleudefrance}{rgb}{0.19, 0.55, 0.91}
\definecolor{bostonuniversityred}{rgb}{0.8, 0.0, 0.0}
\definecolor{darkred}{rgb}{0.55, 0.0, 0.0}
\definecolor{blue(ryb)}{rgb}{0.01, 0.28, 1.0}
\definecolor{darkgreen}{rgb}{0.0, 0.5, 0.0}
\definecolor{palatinatepurple}{rgb}{0.41, 0.16, 0.38}
\definecolor{cadmiumorange}{rgb}{0.93, 0.53, 0.18}
\definecolor{airforceblue}{rgb}{0.36, 0.54, 0.66}
\definecolor{ceruleanblue}{rgb}{0.16, 0.32, 0.75}
\definecolor{applegreen}{rgb}{0.55, 0.71, 0.0}
\definecolor{antiquebrass}{rgb}{0.8, 0.58, 0.46}
\definecolor{cobalt}{rgb}{0.0, 0.28, 0.67}
\definecolor{darkorchid}{rgb}{0.6, 0.2, 0.8}
\definecolor{darkpastelgreen}{rgb}{0.01, 0.75, 0.24}
\definecolor{darkpastelred}{rgb}{0.76, 0.23, 0.13}
\definecolor{darkspringgreen}{rgb}{0.09, 0.45, 0.27}

%\def\KstarP  {\ensuremath{\kaon^{*}(892)^{0}}\xspace}
\def\KstarP  {\ensuremath{\kaon^{*}(892)}\xspace}
%\def\Kstarfourteenthirty  {{\ensuremath{\kaon^{*}_{0,2}(1430)^{0}}}\xspace}
\def\Kstarfourteenthirty  {{\ensuremath{\kaon^{*}(1430)}}\xspace}
\DeclareRobustCommand{\orderof}{\ensuremath{\mathcal{O}}}
\usepackage[export]{adjustbox}

\tikzset{
  every overlay node/.style={
    draw=white,anchor=north west,
  },
}
\def\tikzoverlay{%
   \tikz[baseline,overlay]\node[every overlay node]
}%

\renewcommand\mathfamilydefault{\rmdefault}

%%%%%%%%%%%%%%%%%%%%%%%%%%%%%%%%%%%%%%%%%%%%%%%%%%%%%%%%%%%%%%%%%
%Begin document
%%%%%%%%%%%%%%%%%%%%%%%%%%%%%%%%%%%%%%%%%%%%%%%%%%%%%%%%%%%%%%%%%
\begin{document}
\title[]{Upstream Tracking and the Decay $\B^{0} \to K^{+}\pi^{-}\mu^{+}\mu^{-}$\\ at the LHCb Experiment}
\author[Espen Eie Bowen]{{\bf Espen Eie Bowen} \\ {\small PhD defense presentation}} 
\institute[]{}
\date{\nth{26} January 2017}
%%%%%%%%%%%%%%%%%%%%%%%%%%%%%%%%%%%%%%%%%%%%%%%%%%%%%%%%%%%%%%%%%
%Title
%%%%%%%%%%%%%%%%%%%%%%%%%%%%%%%%%%%%%%%%%%%%%%%%%%%%%%%%%%%%%%%%%
\begin{frame}[plain]
\titlepage
\begin{textblock*}{2cm}(12.5cm,0.0cm)
  \includegraphics[width=2cm]{figs/lhcb-logo.pdf}
\end{textblock*}
\begin{textblock*}{5cm}(0.0cm,0.0cm)
  \includegraphics[width=3.0cm]{figs/uzh.jpg}
\end{textblock*}
\end{frame}

%%%%%%%%%%%%%%%%%%%%%%%%%%%%%%%%%%%%%%%%%%%%%%%%%%%%%%%%%%%%%%%%%
%Slide
%%%%%%%%%%%%%%%%%%%%%%%%%%%%%%%%%%%%%%%%%%%%%%%%%%%%%%%%%%%%%%%%%
\begin{frame}{The Standard Model of particle physics}
\begin{columns}
\begin{column}{0.5\textwidth}
\includegraphics[width=\textwidth]{figs/theory/SM.pdf}
\end{column}
\begin{column}{0.5\textwidth}
\begin{itemize}
\item The Standard Model describes our current understanding of elementary particles and their interactions
\item[]
\item All particles predicted by the Standard Model have been observed experimentally
\item[]
\item It has been very successful in predicting cross-sections of interaction processes with great precision
\end{itemize}
\end{column}
\end{columns}
\end{frame}

%%%%%%%%%%%%%%%%%%%%%%%%%%%%%%%%%%%%%%%%%%%%%%%%%%%%%%%%%%%%%%%%%
%Slide
%%%%%%%%%%%%%%%%%%%%%%%%%%%%%%%%%%%%%%%%%%%%%%%%%%%%%%%%%%%%%%%%%
\begin{frame}{Not the whole story...}

\begin{center}
\includegraphics[height=0.5\textheight]{figs/matter-antimatter.jpg}
\includegraphics[height=0.5\textheight]{figs/darkmatter.jpg}
\includegraphics[height=0.5\textheight]{figs/13_oscill.jpeg}
\end{center}

\begin{itemize}
  \item There is an observed asymmetry between matter and antimatter in our Universe
  \item Dark matter, which is believed to be 5 times as abundant as ordinary matter, is not accounted for
  \item SM neutrinos are only left-handed and massless, but we know they have a small but non-zero mass
\end{itemize}
\end{frame}

%%%%%%%%%%%%%%%%%%%%%%%%%%%%%%%%%%%%%%%%%%%%%%%%%%%%%%%%%%%%%%%%%
%Slide
%%%%%%%%%%%%%%%%%%%%%%%%%%%%%%%%%%%%%%%%%%%%%%%%%%%%%%%%%%%%%%%%%
\begin{frame}{New Physics}
\begin{itemize}
\item These limitations have led particle physicists to look for extensions to the SM in the form of New Physics (NP) models
\end{itemize}

\begin{columns}
\begin{column}{0.5\textwidth}
\begin{itemize}
  \item e.g. Supersymmetry (SUSY)
\begin{itemize}
  \item Doubles the amount of particles
  \item Introduces fermionic partner to every boson and bosonic partner to every fermion
  \item Lightest supersymmetric particle provides dark matter candidate
\end{itemize}
\end{itemize}
\end{column}
\begin{column}{0.5\textwidth}
\includegraphics[width=\textwidth]{figs/supersymmetry.jpg}
\end{column}
\end{columns}

\bigskip

\begin{itemize}
\item Can search for these NP particles through {\bf direct} and {\bf indirect} approaches
   \begin{itemize}
     \item Direct \to produce new particles that can be directly detected
     \item Indirect \to study the effect of `virtual' particles within quantum loops
   \end{itemize}
\end{itemize}

\end{frame}

%%%%%%%%%%%%%%%%%%%%%%%%%%%%%%%%%%%%%%%%%%%%%%%%%%%%%%%%%%%%%%%%%
%Slide
%%%%%%%%%%%%%%%%%%%%%%%%%%%%%%%%%%%%%%%%%%%%%%%%%%%%%%%%%%%%%%%%%
% \begin{frame}{Direct searches for NP}

% \end{frame}

%%%%%%%%%%%%%%%%%%%%%%%%%%%%%%%%%%%%%%%%%%%%%%%%%%%%%%%%%%%%%%%%%
%Slide
%%%%%%%%%%%%%%%%%%%%%%%%%%%%%%%%%%%%%%%%%%%%%%%%%%%%%%%%%%%%%%%%%
\begin{frame}{Rare decays as indirect searches for NP}
\vspace{-0.2cm}
\begin{columns}
\begin{column}{0.5\textwidth}
\begin{myenv}{\color{bleudefrance}{SM}}[linecolor=bleudefrance]
\centering
\begin{tikzpicture}
\node[anchor=south west,inner sep=0](image) at (0,0) {\includegraphics[width=0.9\textwidth]{figs/Bsmm-1.pdf}};
\begin{scope}[x={(image.south east)},y={(image.north west)}]
%\draw[help lines,xstep=.1,ystep=.1] (0,0) grid (1,1);
% \node[draw=none,bleudefrance] at (0.53,0.98) {\scriptsize \Wm};
% \node[draw=none,bleudefrance] at (0.3,0.6) {\scriptsize \tquark};
% \node[draw=none,bleudefrance] at (0.56,0.27) {\scriptsize {\small\Pgamma},$\Z^{0}$};
\end{scope}
\end{tikzpicture}
\end{myenv}
\end{column}
\begin{column}{0.5\textwidth}
\begin{myenv}{\color{bostonuniversityred}{NP}}[linecolor=bostonuniversityred]
\centering
\begin{tikzpicture}
\centering
\node[anchor=south west,inner sep=0](image) at (0,0) {\includegraphics[width=0.9\textwidth]{figs/Bsmm-2.pdf}};
      \begin{scope}[x={(image.south east)},y={(image.north west)}]
      %\draw[help lines,xstep=.1,ystep=.1] (0,0) grid (1,1);
      % \node[draw=none,bostonuniversityred] at (0.5,0.96) {\small ?};
      % \node[draw=none,bostonuniversityred] at (0.3,0.6) {\small ?};
      % \node[draw=none,bostonuniversityred] at (0.58,0.27) {\small ?};
      \end{scope}
\end{tikzpicture}
\end{myenv}
\end{column}
\end{columns}

\bigskip

\begin{itemize}
\item Decays such as \Bsmm proceed via loop level processes
\item Virtual particles can ``borrow'' energy for a short time due to the Heisenberg uncertainty principle
\item If new particles exist, they can also appear in the loop
\item As decays such as \Bsmm are suppressed in the SM, NP can contribute at the same level and change the decay properties
\end{itemize}

\end{frame}

%%%%%%%%%%%%%%%%%%%%%%%%%%%%%%%%%%%%%%%%%%%%%%%%%%%%%%%%%%%%%%%%%
%Slide
%%%%%%%%%%%%%%%%%%%%%%%%%%%%%%%%%%%%%%%%%%%%%%%%%%%%%%%%%%%%%%%%%
\begin{frame}{The \lhcb experiment}
  \begin{itemize}
  \item LHCb is the dedicated heavy flavour physics experiment at the LHC 
  \item Its primary goal is to search for {\bf indirect} evidence of New Physics in \CP violation and rare decays of beauty and charm hadrons
  \item Takes advantage of large production of \bquark\bquarkbar pairs in the \proton\proton collisions
  \end{itemize}

   \bigskip
    
   \begin{columns}
   \begin{column}{0.4\textwidth}
    \begin{itemize}
      \item Single arm forward spectrometer
      \item Excellent particle identification \mbox{(e.g. $\pi \to \mu <0.5\%$)}
      \item Excellent momentum resolution \mbox{(e.g. $\delta\ptot/\ptot \sim 0.5\%$)}
    \end{itemize}
    \end{column}
   \begin{column}{0.6\textwidth}
      \includegraphics[width=\textwidth]{figs/lhcb/lhcb2.png}
    \end{column}
   \end{columns}
\end{frame}

%%%%%%%%%%%%%%%%%%%%%%%%%%%%%%%%%%%%%%%%%%%%%%%%%%%%%%%%%%%%%%%%%
%Slide
%%%%%%%%%%%%%%%%%%%%%%%%%%%%%%%%%%%%%%%%%%%%%%%%%%%%%%%%%%%%%%%%%
\begin{frame}\frametitle{}
  \begin{center}
    \begin{tikzpicture}
      \node[anchor=south west,inner sep=0](image) at (0,0) {\includegraphics[width=\textwidth]{figs/lhcb/lhcb2.png}};
      
      \begin{scope}[x={(image.south east)},y={(image.north west)}]
        %\draw[help lines,xstep=.1,ystep=.1] (0,0) grid (1,1);
        
        \node[draw=none,burgundy] at (0.17,0.05) {Vertex Locator (\velo)};
        \draw[ultra thick,->,burgundy] (0.15,0.1) -- (0.2,0.4);
        
        \node[draw=none,cadmiumorange] at (0.20,0.95) {Tracking system (TT, IT, OT)};
        \draw[ultra thick,->,cadmiumorange] (0.14,0.9) -- (0.3,0.48);
        \draw[ultra thick,->,cadmiumorange] (0.14,0.9) -- (0.58,0.52); 
        
        \node[draw=none,byzantine] at (0.49,0.95) {RICH detectors};
        \draw[ultra thick,->,byzantine] (0.5,0.9) -- (0.23,0.55);
        \draw[ultra thick,->,byzantine] (0.5,0.9) -- (0.65,0.65);
        
        \node[draw=none,gray] at (0.45,0.2) {Magnet};
        
        \node[draw=none,babyblue] at (0.7,0.05) {Calorimeters};
        \draw[ultra thick,->,babyblue] (0.7,0.1) -- (0.75,0.55);
        
        \node[draw=none,green!90!black] at (0.9,0.7) {Muon system};
        
      \end{scope}
    \end{tikzpicture}
    
  \end{center}
\end{frame}

%%%%%%%%%%%%%%%%%%%%%%%%%%%%%%%%%%%%%%%%%%%%%%%%%%%%%%%%%%%%%%%%%
%Slide
%%%%%%%%%%%%%%%%%%%%%%%%%%%%%%%%%%%%%%%%%%%%%%%%%%%%%%%%%%%%%%%%%
\begin{frame}{Track reconstruction at \lhcb}

\begin{columns}
\begin{column}{0.5\textwidth}
\begin{itemize}
\item Track reconstruction is used to determine the momentum of charged particles
\end{itemize}

\begin{itemize}
\item At \lhcb this is performed by several different software algorithms
\end{itemize}
\begin{itemize}
\item Two standalone algorithms 
  \begin{itemize}
    \item[\ding{70}] \velo tracking and T track seeding
  \end{itemize}
\item Other algorithms use input from these to perform further track reconstruction
  \begin{itemize}
    \item[\ding{70}] e.g. Forward tracking uses \velo tracks
  \end{itemize}
\end{itemize}
\end{column}
\begin{column}{0.5\textwidth}
\begin{tikzpicture}
\node[anchor=south west,inner sep=0](image) at (0,0) {\includegraphics[trim={1cm 0 2cm 0},clip,width=\textwidth]{figs/tracking/trackTypes.pdf}};
\begin{scope}[x={(image.south east)},y={(image.north west)}]
%\draw[help lines,xstep=.1,ystep=.1] (0,0) grid (1,1);
\fill[white] (0.22,0.58) rectangle (0.29,0.62);
\node at (0.25,0.6) {\footnotesize {\fontfamily{cmss}\selectfont TT}};
\end{scope}
\end{tikzpicture} 
\end{column}
\end{columns}
\end{frame}

%%%%%%%%%%%%%%%%%%%%%%%%%%%%%%%%%%%%%%%%%%%%%%%%%%%%%%%%%%%%%%%%%
%Slide
%%%%%%%%%%%%%%%%%%%%%%%%%%%%%%%%%%%%%%%%%%%%%%%%%%%%%%%%%%%%%%%%%
\begin{frame}{The \lhcb trigger in Run 1 (2010-2013)}

\begin{itemize}
  \item The data rate from every \proton\proton collision is too high for all of it to be stored
  \item Moreover, many of the collisions are not of interest for physics analyses
  \item The \lhcb trigger plays important role in selecting interesting events and reducing the data rate to disk 
\end{itemize}

\begin{itemize}
 \item During Run I, this consisted of two stages:
 \begin{enumerate}
    \item Hardware trigger
    \begin{itemize}
      \item Decision based on information from muon system and calorimeters 
      \item Reduces data rate from $40\mhz\rightarrow1\mhz$
    \end{itemize}
    \item Software trigger 
    \begin{itemize}
      \item Track reconstruction performed
      \item Decision based on multivariate classifier
      \item Reduces data rate from 1\mhz $\rightarrow$ 5\khz, which is stored for offline analysis
    \end{itemize}
 \end{enumerate}
\end{itemize}

\end{frame}

%%%%%%%%%%%%%%%%%%%%%%%%%%%%%%%%%%%%%%%%%%%%%%%%%%%%%%%%%%%%%%%%%
%Slide
%%%%%%%%%%%%%%%%%%%%%%%%%%%%%%%%%%%%%%%%%%%%%%%%%%%%%%%%%%%%%%%%%
\begin{frame}{The \lhcb Upgrade (2020+)}

\begin{columns}
\begin{column}{0.6\textwidth}
  \begin{itemize}
    \item \lhcb  will undergo upgrade during 2018-2019 to allow data taking at higher energy and intensity
    \item Many existing subdetectors will be replaced
    \begin{itemize}
      \item[\ding{80}] e.g. TT replaced by UT
    \end{itemize}
  \end{itemize}

  \begin{itemize}
    \item Key feature: Full software trigger
    \begin{itemize}
      \item[\ding{70}] Remove hardware trigger stage
      \item[\ding{70}] Improved trigger efficiency for many physics channels
      \item[\ding{80}] Strict requirements placed on execution time of tracking algorithms
      \item[\ding{80}] Existing tracking sequence (\velo \to Forward) {\bf unable to meet timing budget!}
    \end{itemize}
    %     \item[\ding{70}] Read out full detector at 40\mhz
    %     \item[\ding{80}] Front end electronics need to be replaced
    %     \item[\ding{80}] Many existing sub-detectors will be upgraded
    %     %\begin{itemize}
    %     %  \item[\ding{80}] e.g. TT replaced by UT
    %     %\end{itemize}
    %   \end{itemize}
    % \begin{enumerate}
    %   \item Triggerless readout
    %   \begin{itemize}
    %     \item[\ding{70}] Remove hardware trigger 
    %     \item[\ding{70}] Read out full detector at 40\mhz
    %     \item[\ding{80}] Front end electronics need to be replaced
    %     \item[\ding{80}] Many existing sub-detectors will be upgraded
    %     %\begin{itemize}
    %     %  \item[\ding{80}] e.g. TT replaced by UT
    %     %\end{itemize}
    %   \end{itemize}
    %   \item Full software trigger
    %   \begin{itemize}
    %     \item[\ding{70}] Full event reconstruction
    %     \item[\ding{70}] Offers great flexibility for designing selections
    %     \item[\ding{70}] Increases trigger efficiency for many physics channels
    %     \item[\ding{80}] Strict requirements on execution time of tracking algorithms
    %     \item[\ding{80}] Existing tracking sequence {\bf unable to meet timing budget!}
    %   \end{itemize}
    % \end{enumerate}
  \end{itemize}

  % \begin{itemize}
  %   \item Front end electronics need to be replaced
  %   \item Many existing sub-detectors will be upgraded
  %   \begin{itemize}
  %       \item[\ding{70}] e.g. TT replaced by UT
  %     \end{itemize}
  % \end{itemize}

\end{column}
\begin{column}{0.4\textwidth}
\centering
\includegraphics[width=0.9\textwidth]{figs/detector/UT.pdf}\\~\\~\\
\begin{tikzpicture}
      \node[anchor=south west,inner sep=0](image) at (0,0) {\includegraphics[width=0.9\textwidth]{figs/need-for-speed.png}};
      \begin{scope}[x={(image.south east)},y={(image.north west)}]
      %\draw[help lines,xstep=.1,ystep=.1] (0,0) grid (1,1);
      \node[draw=none] at (0.65,-0.1) {\bf LHCb Upgrade};
      %\node[draw=none,bleudefrance] at (0.3,0.6) {\scriptsize \tquark};
      %\node[draw=none,bleudefrance] at (0.56,0.27) {\scriptsize {\small\Pgamma},$\Z^{0}$};
      \end{scope}
      \end{tikzpicture}
%\includegraphics[width=0.9\textwidth]{figs/need-for-speed.png}
\end{column}

\end{columns}

\end{frame}

% %%%%%%%%%%%%%%%%%%%%%%%%%%%%%%%%%%%%%%%%%%%%%%%%%%%%%%%%%%%%%%%%%
% %Slide
% %%%%%%%%%%%%%%%%%%%%%%%%%%%%%%%%%%%%%%%%%%%%%%%%%%%%%%%%%%%%%%%%%
% \begin{frame}{The \lhcb Upgrade (2020+)}

% \begin{columns}
% \begin{column}{0.6\textwidth}
%   \begin{itemize}
%     \item \lhcb  will undergo upgrade during 2018-2019 to allow data taking at \mbox{$\sqs=14\tev$ and $\lum=2\times10^{33}\cm^{-2}\sec^{-1}$}
%   \end{itemize}

%   \begin{itemize}
%     \item Two key features
%     \begin{enumerate}
%       \item Triggerless readout
%       \begin{itemize}
%         \item[\ding{70}] Remove L0 trigger bottleneck
%         \item[\ding{70}] Read out full detector at 40\mhz
%         \item[\ding{80}] Front end electronics need to be replaced
%         \item[\ding{80}] Many existing sub-detectors will be upgraded
%         %\begin{itemize}
%         %  \item[\ding{80}] e.g. TT replaced by UT
%         %\end{itemize}
%       \end{itemize}
%       \item Full software trigger
%       \begin{itemize}
%         \item[\ding{70}] Full event reconstruction
%         \item[\ding{70}] Offers great flexibility for designing selections
%         \item[\ding{70}] Increases trigger efficiency for many physics channels
%         \item[\ding{80}] Strict requirements on execution time of tracking algorithms
%         \item[\ding{80}] Existing tracking sequence {\bf unable to meet timing budget!}
%       \end{itemize}
%     \end{enumerate}
%   \end{itemize}

%   % \begin{itemize}
%   %   \item Front end electronics need to be replaced
%   %   \item Many existing sub-detectors will be upgraded
%   %   \begin{itemize}
%   %       \item[\ding{70}] e.g. TT replaced by UT
%   %     \end{itemize}
%   % \end{itemize}

% \end{column}
% \begin{column}{0.4\textwidth}
% \centering
% \vspace{-1cm}
% \adjincludegraphics[width=\textwidth,trim={{.3\width} {.575\height} {.28\width} {.1\height}},clip]{figs/detector/upgrade-motivation.pdf}
% \includegraphics[width=0.8\textwidth]{figs/detector/UT.pdf}
% \end{column}

% \end{columns}

% \end{frame}

%%%%%%%%%%%%%%%%%%%%%%%%%%%%%%%%%%%%%%%%%%%%%%%%%%%%%%%%%%%%%%%%%
%Slide
%%%%%%%%%%%%%%%%%%%%%%%%%%%%%%%%%%%%%%%%%%%%%%%%%%%%%%%%%%%%%%%%%
\begin{frame}{Novel idea}
\begin{itemize}
  \item Can using upstream tracks speed up the full tracking sequence? \mbox{(\velo $\to$ [Upstream] $\to$ Forward)}
\end{itemize}
\begin{itemize}
  \item Advantages over using VELO tracks
    \begin{itemize}
      \item Charge and momentum of track segment ($\delta \ptot/\ptot \sim 15 \%$)
      \begin{itemize}
        \item[\ding{212}] Preselect on track \pt
        \item[\ding{212}] Open smaller search windows
        %\item[\ding{70}] Greatly reduce execution time and ghost rate!
        \item[\ding{70}] Greatly reduce execution time!
      \end{itemize}
    \end{itemize}
  \end{itemize}

  \begin{center}
  \input{figs/tikz/upstream-tracking-upgrade-search-window}
  \end{center}

  \begin{itemize}
    \item Upstream tracking algorithm needs to be both fast and efficient
  \end{itemize}
\end{frame}

%%%%%%%%%%%%%%%%%%%%%%%%%%%%%%%%%%%%%%%%%%%%%%%%%%%%%%%%%%%%%%%%%
%Slide
%%%%%%%%%%%%%%%%%%%%%%%%%%%%%%%%%%%%%%%%%%%%%%%%%%%%%%%%%%%%%%%%%
\begin{frame}\frametitle{Upstream tracking (VeloTT) in Run 1}
  \begin{itemize}
  \item[$\blacktriangleright$] Linearly extrapolate VELO track to TT
  \item[$\blacktriangleright$] Open search windows in each layer: calculate $\Delta x$ between track and hits
  \item[$\blacktriangleright$] Scale $\Delta x$ to reference plane at center of TT
  \item[$\blacktriangleright$] Choose clusters of hits consistent with coming from the same VELO track
  \begin{itemize}
    \item[\ding{80}] Each hit must be on the same side of the linear extrapolation
  \end{itemize}
  \item[$\blacktriangleright$] Fit each track candidate with a simple $\chi^{2}$ fit
  \item[$\blacktriangleright$] Choose best candidate track based on \mbox{\# layers} fired and $\chi^{2}$
  \item[\ding{80}] Fit track with full Kalman filter (slow)
  \end{itemize}
  \input{figs/tikz/tracking-velott}
\end{frame}

%%%%%%%%%%%%%%%%%%%%%%%%%%%%%%%%%%%%%%%%%%%%%%%%%%%%%%%%%%%%%%%%%
%Slide
%%%%%%%%%%%%%%%%%%%%%%%%%%%%%%%%%%%%%%%%%%%%%%%%%%%%%%%%%%%%%%%%%
\begin{frame}{Track reconstruction: Performance}

\begin{itemize}
  \item Three important figures of merit used to evaluate the performance of tracking algorithms
\end{itemize}
\begin{itemize}
  \item[\ding{182}] Reconstruction efficiency
  \begin{itemize}
    \item[\ding{70}] N(reconstructible \& reconstructed)/N(reconstructible)
    \item[\ding{70}] Reconstructible means having left sufficient hits in the tracking stations
    \item[\ding{70}] Determined using simulation
  \end{itemize}
\end{itemize}
\begin{itemize}
  \item[\ding{183}] Ghost rate
  \begin{itemize}
  \item[\ding{70}] N(ghost tracks)/N(tracks)
  \item[\ding{70}] A ghost track is one which has no matching simulated particle
  \item[\ding{70}] Determined using simulation
  \end{itemize}
\end{itemize}
\begin{itemize}
  \item[\ding{184}] Average execution time per event
\end{itemize}

\end{frame}
%%%%%%%%%%%%%%%%%%%%%%%%%%%%%%%%%%%%%%%%%%%%%%%%%%%%%%%%%%%%%%%%%
%Slide
%%%%%%%%%%%%%%%%%%%%%%%%%%%%%%%%%%%%%%%%%%%%%%%%%%%%%%%%%%%%%%%%%
\begin{frame}{VeloUT: Initial performance}

\begin{columns}
\begin{column}{0.65\textwidth}

\begin{itemize}
  \item Initial version of VeloUT a replication of VeloTT algorithm from Run 1
\end{itemize}

\begin{itemize}
  \item Execution time of 27.20\ms ~{\bf far too slow} to use in the software trigger
  \begin{itemize}
    \item[\ding{70}] c.f. VELO tracking $\sim1.8\ms$
  \end{itemize}
  \item Reconstruction efficiency {\bf too low}
  \begin{itemize}
    \item[\ding{70}] Any inefficiency will be propagated to next step
    \item[\ding{70}] Decreases as a function of momentum
  \end{itemize}
\end{itemize}

\bigskip

\begin{mdframed}[linecolor=barcolor]
\begin{center}
\resizebox{\columnwidth}{!}{
\begin{tabular}{c|c|c|c}
  \velout & Efficiency [\%] & Ghost rate [\%] & Timing [ms] \\ 
  \hline
  v1r2  & 93.94  & 7.21  &  27.20  \\
 \end{tabular}
 }
\end{center}
\end{mdframed}
\end{column}

\begin{column}{0.35\textwidth}
\centering
\begin{figure}
\vspace*{-1cm}
\includegraphics[height=0.475\textheight]{figs/upstream-tracking-upgrade/eff_p_v1r2.pdf}\\
\includegraphics[height=0.475\textheight]{figs/upstream-tracking-upgrade/gr_p_v1r2.pdf}
\end{figure}
\end{column}
\end{columns}

\end{frame}

%%%%%%%%%%%%%%%%%%%%%%%%%%%%%%%%%%%%%%%%%%%%%%%%%%%%%%%%%%%%%%%%%
%Slide
%%%%%%%%%%%%%%%%%%%%%%%%%%%%%%%%%%%%%%%%%%%%%%%%%%%%%%%%%%%%%%%%%
\begin{frame}{VeloUT: Improvements}

\begin{myenv}{Execution time much too slow for use in trigger}[linecolor=barcolor]
  \begin{itemize}
    \setlength{\itemindent}{0.5em}
    \item[\ding{70}] Introduced hit sorting and binary searches
    \item[\ding{70}] Removed Kalman filter, simple $\chi^{2}$ fit is sufficient
    \item[\ding{70}] Numerous \cpp optimisations/changes to the logic
  \end{itemize}
\end{myenv}

\begin{myenv}{Efficiency too low/falls off with increasing momentum}[linecolor=barcolor]
  \begin{itemize}
    \setlength{\itemindent}{0.5em}
    \item[$\blacktriangleright$] Tracked down to poor assumptions made when associating hits to \velo tracks
    \item[\ding{70}] Developed new hit clustering sequence 
  \end{itemize}
\end{myenv}

\end{frame}

%%%%%%%%%%%%%%%%%%%%%%%%%%%%%%%%%%%%%%%%%%%%%%%%%%%%%%%%%%%%%%%%%
%Slide
%%%%%%%%%%%%%%%%%%%%%%%%%%%%%%%%%%%%%%%%%%%%%%%%%%%%%%%%%%%%%%%%%
\begin{frame}{VeloUT: Optimised algorithm}
\begin{columns}
\begin{column}{0.5\textwidth}
\begin{itemize}
  \item[$\blacktriangleright$] Linearly extrapolate VELO track to UT
  \item[$\blacktriangleright$] Select hits within a search window around the extrapolated track
  \item[\ding{80}] Form doublets of hits in the first two layers
  \item[\ding{80}] Extrapolate doublets to third/fourth layers and search for compatible hits
  \item[\ding{80}] If no four hit candidates found, repeat starting from last two layers
  \item[$\blacktriangleright$] Fit each track candidate with a simple $\chi^{2}$ fit% and estimate $q/p$ ($\delta p/p \sim 15 \%$)
  \item[$\blacktriangleright$] Choose best candidate track based on \mbox{\# layers} fired and $\chi^{2}$
  \end{itemize}
\end{column}
\begin{column}{0.5\textwidth}
\centering
\input{figs/tikz/upstream-tracking-upgrade-clustering}
\end{column}
\end{columns}
\end{frame}

%%%%%%%%%%%%%%%%%%%%%%%%%%%%%%%%%%%%%%%%%%%%%%%%%%%%%%%%%%%%%%%%%
%Slide
%%%%%%%%%%%%%%%%%%%%%%%%%%%%%%%%%%%%%%%%%%%%%%%%%%%%%%%%%%%%%%%%%
\begin{frame}{VeloUT: Optimised peformance}

\begin{columns}
\begin{column}{0.65\textwidth}
\begin{itemize}
  \item Huge reduction in the execution time ($\times$33!)
  \item Large improvement in the efficiency ($+5$\%)
  \begin{itemize}
    \item[\ding{70}] Now flat in \ptot
  \end{itemize}
  \item Slight increase in the ghost rate ($+0.8$\%)
  \begin{itemize}
    \item[\ding{70}] Can be reduced in offline analysis
  \end{itemize}
\end{itemize}

\bigskip

\begin{mdframed}[linecolor=barcolor]
\begin{center}
\resizebox{\columnwidth}{!}{
\begin{tabular}{c|c|c|c}
  \velout & Efficiency [\%] & Ghost rate [\%] & Timing [ms] \\ 
  \hline
  v1r2  & 93.94  & 7.21  &  27.20  \\
  v2r2  & 98.69  & 8.00 &  \hphantom{0}0.81  \\
 \end{tabular}
 }
\end{center}
\end{mdframed}
\end{column}

\begin{column}{0.35\textwidth}
\centering
\begin{figure}
\vspace*{-1cm}
\includegraphics[height=0.475\textheight]{figs/upstream-tracking-upgrade/eff_p_comp.pdf}\\
\includegraphics[height=0.475\textheight]{figs/upstream-tracking-upgrade/gr_p_comp.pdf}
\end{figure}
\end{column}
\end{columns}

\end{frame}

%%%%%%%%%%%%%%%%%%%%%%%%%%%%%%%%%%%%%%%%%%%%%%%%%%%%%%%%%%%%%%%%%
%Slide
%%%%%%%%%%%%%%%%%%%%%%%%%%%%%%%%%%%%%%%%%%%%%%%%%%%%%%%%%%%%%%%%%
\begin{frame}{VeloUT-Forward: Optimised peformance}

\begin{columns}
\begin{column}{0.65\textwidth}

\begin{itemize}
  \item[] Most interested in effect on full tracking sequence (\velo $\to$ [Upstream] $\to$ Forward)
\end{itemize}

\begin{itemize}
  \item Significant reduction in the execution time ($\times$4)
  \item Large reduction in the ghost rate ($\times$3)
  \item Some loss of efficiency ($-0.7$\%)
\end{itemize}

\bigskip

\begin{mdframed}[linecolor=barcolor]
\begin{center}
\resizebox{\columnwidth}{!}{
\begin{tabular}{c|c|c|c}
    & Efficiency [\%] & Ghost rate [\%] & Timing [ms] \\
   \hline
   Velo-Fwd  & 94.10  & 41.55  &  18.28 \\
   VeloUT-Fwd  & 93.37  & 14.08  &  0.81+3.45  \\
 \end{tabular}
 }
\end{center}
\end{mdframed}
\end{column}

\begin{column}{0.35\textwidth}
\centering
\begin{figure}
\vspace*{-1cm}
\includegraphics[height=0.475\textheight]{figs/upstream-tracking-upgrade/fwd_eff_p_comp.pdf}\\
\includegraphics[height=0.475\textheight]{figs/upstream-tracking-upgrade/fwd_gr_p_comp.pdf}
\end{figure}
\end{column}
\end{columns}

\end{frame}

%%%%%%%%%%%%%%%%%%%%%%%%%%%%%%%%%%%%%%%%%%%%%%%%%%%%%%%%%%%%%%%%%
%Slide
%%%%%%%%%%%%%%%%%%%%%%%%%%%%%%%%%%%%%%%%%%%%%%%%%%%%%%%%%%%%%%%%%
\begin{frame}{Upstream tracking for the Upgrade: Summary}

\begin{columns}
\begin{column}{0.75\textwidth}
\begin{itemize}
  \item[$\blacktriangleright$] Vast improvements in the performance of the \velout algorithm
  \item[$\blacktriangleright$] Leads to subsequent improvement to the overall tracking sequence
  \item[\ding{80}] Adopted into the default tracking sequence for the \lhcb Upgrade
\end{itemize}
\end{column}
\begin{column}{0.25\textwidth}
%\centering
\href{https://cds.cern.ch/record/1624070}{\includegraphics[width=0.48\textwidth]{figs/VELO.pdf}}
\href{https://cds.cern.ch/record/1624074}{\includegraphics[width=0.48\textwidth]{figs/PID.pdf}}\\
\href{https://cds.cern.ch/record/1647400}{\includegraphics[width=0.48\textwidth]{figs/Tracking.pdf}}
\href{https://cds.cern.ch/record/1701361}{\includegraphics[width=0.48\textwidth]{figs/Trigger.pdf}}
\end{column}
\end{columns}

\bigskip\bigskip

\begin{itemize}
\item[\ding{80}] \lhcb will become first hadron collider experiment to operate a software-only trigger
\end{itemize}

\end{frame}

%%%%%%%%%%%%%%%%%%%%%%%%%%%%%%%%%%%%%%%%%%%%%%%%%%%%%%%%%%%%%%%%%
%Slide
%%%%%%%%%%%%%%%%%%%%%%%%%%%%%%%%%%%%%%%%%%%%%%%%%%%%%%%%%%%%%%%%%
{\usebackgroundtemplate{\includegraphics[width=1.05\paperwidth]{figs/cavern.jpg}}
 
 \begin{frame}[plain]
 \vspace{8.75cm}
 \hspace{-0.75cm}
 \huge\color{anti-flashwhite}{The decay \BdToKpimm}
 \end{frame}
}

%%%%%%%%%%%%%%%%%%%%%%%%%%%%%%%%%%%%%%%%%%%%%%%%%%%%%%%%%%%%%%%%%
%Slide
%%%%%%%%%%%%%%%%%%%%%%%%%%%%%%%%%%%%%%%%%%%%%%%%%%%%%%%%%%%%%%%%%
\begin{frame}{Reminder: Rare decays as a probe for NP}
  \begin{columns}
    \begin{column}{0.6\textwidth}
      \begin{itemize}
      \item The decay \BdToKpimm is an example of a \btosll flavour-changing-neutral-current processes that proceeds via loop diagrams in SM
        % \begin{itemize}
        % \item Highly suppressed
        % \end{itemize}
      \item New, heavy particles in SM extensions can enter the loop and modify observables 
      \begin{itemize}
        \item e.g. enhance/suppress branching fractions, alter angular distributions, new sources of \CP violation
      \end{itemize}
      \end{itemize}
    \end{column}
    
    \begin{column}{0.4\textwidth}
      \vspace{-1cm}
      \begin{myenv}{\color{bleudefrance}{SM}}[linecolor=bleudefrance]
      \centering
      \begin{tikzpicture}
      \node[anchor=south west,inner sep=0](image) at (0,0) {\includegraphics[width=0.7\textwidth]{figs/theory/btosll_penguin.eps}};
      \begin{scope}[x={(image.south east)},y={(image.north west)}]
      %\draw[help lines,xstep=.1,ystep=.1] (0,0) grid (1,1);
      \node[draw=none,bleudefrance] at (0.53,0.98) {\scriptsize \Wm};
      \node[draw=none,bleudefrance] at (0.3,0.6) {\scriptsize \tquark};
      \node[draw=none,bleudefrance] at (0.56,0.27) {\scriptsize {\small\Pgamma},$\Z^{0}$};
      \end{scope}
      \end{tikzpicture}
      \begin{tikzpicture}
      \node[anchor=south west,inner sep=0](image) at (0,0) {\includegraphics[width=0.7\textwidth]{figs/theory/btosll_box.eps}};
      \begin{scope}[x={(image.south east)},y={(image.north west)}]
      %\draw[help lines,xstep=.1,ystep=.1] (0,0) grid (1,1);
      \node[draw=none,bleudefrance] at (0.5,0.95) {\scriptsize \tquark};
      \node[draw=none,bleudefrance] at (0.23,0.6) {\scriptsize \Wm};
      \node[draw=none,bleudefrance] at (0.8,0.6) {\scriptsize \Wp};
      \node[draw=none,bleudefrance] at (0.5,0.45) {\scriptsize \Pnu};
      \end{scope}
      \end{tikzpicture}
      \end{myenv}
    \end{column}
    \end{columns}

    \begin{myenv}{\color{bostonuniversityred}{NP}}[linecolor=bostonuniversityred]
      \centering
      \begin{tikzpicture}
      \node[anchor=south west,inner sep=0](image) at (0,0) {\includegraphics[width=0.3\textwidth]{figs/theory/btosll_penguin.eps}};
      \begin{scope}[x={(image.south east)},y={(image.north west)}]
      %\draw[help lines,xstep=.1,ystep=.1] (0,0) grid (1,1);
      \node[draw=none,bostonuniversityred] at (0.5,0.96) {\small ?};
      \node[draw=none,bostonuniversityred] at (0.3,0.6) {\small ?};
      \node[draw=none,bostonuniversityred] at (0.58,0.27) {\small ?};
      \end{scope}
      \end{tikzpicture}
      \begin{tikzpicture}
      \node[anchor=south west,inner sep=0](image) at (0,0) {\includegraphics[width=0.3\textwidth]{figs/theory/btosll_box.eps}};
      \begin{scope}[x={(image.south east)},y={(image.north west)}]
      %\draw[help lines,xstep=.1,ystep=.1] (0,0) grid (1,1);
      \node[draw=none,bostonuniversityred] at (0.5,0.97) {\small ?};
      \node[draw=none,bostonuniversityred] at (0.25,0.6) {\small ?};
      \node[draw=none,bostonuniversityred] at (0.8,0.6) {\small ?};
      \node[draw=none,bostonuniversityred] at (0.5,0.49) {\small ?};
      \end{scope}
      \end{tikzpicture}
    \begin{tikzpicture}
      \node[anchor=south west,inner sep=0](image) at (0,0) {\includegraphics[width=0.3\textwidth]{figs/theory/btosll_zprime.eps}};
      \begin{scope}[x={(image.south east)},y={(image.north west)}]
      %\draw[help lines,xstep=.1,ystep=.1] (0,0) grid (1,1);
      \node[draw=none,bostonuniversityred] at (0.55,0.45) {\small ?};
      \end{scope}
      \end{tikzpicture}
    \end{myenv}

\end{frame}

%%%%%%%%%%%%%%%%%%%%%%%%%%%%%%%%%%%%%%%%%%%%%%%%%%%%%%%%%%%%%%%%%
%Slide
%%%%%%%%%%%%%%%%%%%%%%%%%%%%%%%%%%%%%%%%%%%%%%%%%%%%%%%%%%%%%%%%%
\begin{frame}{Previous measurements: Differential branching fractions}

\begin{itemize}
  \item[$\blacktriangleright$] Large LHCb datasets allow for precise measurements of the differential branching fractions of \btosll processes
  \item[\ding{80}] Results point towards lower rates than predicted by theory
\end{itemize}

\begin{center}
\begin{tikzpicture}
 \centering
 \node[anchor=south west,inner sep=0](image) at (0,0) {\includegraphics[height=0.33\textheight]{figs/BFs/BdKstmm.pdf}};
 \begin{scope}[x={(image.south east)},y={(image.north west)}]
  %\draw[help lines,xstep=.1,ystep=.1] (0,0) grid (1,1);
  \node[draw=none] at (0.55,0.98) {\tiny \BdToKstmm};
 \end{scope}
\end{tikzpicture}
\begin{tikzpicture}
 \centering
 \node[anchor=south west,inner sep=0](image) at (0,0) {\includegraphics[height=0.33\textheight]{figs/BFs/BsPhiMM.pdf}};
 \begin{scope}[x={(image.south east)},y={(image.north west)}]
  %\draw[help lines,xstep=.1,ystep=.1] (0,0) grid (1,1);
  \node[draw=none] at (0.55,1.01) {\tiny \BsTophimm};
 \end{scope}
\end{tikzpicture}
\begin{tikzpicture}
 \centering
 \node[anchor=south west,inner sep=0](image) at (0,0) {\includegraphics[height=0.33\textheight]{figs/BFs/LbLMM.pdf}};
 \begin{scope}[x={(image.south east)},y={(image.north west)}]
  %\draw[help lines,xstep=.1,ystep=.1] (0,0) grid (1,1);
  \node[draw=none] at (0.55,1.01) {\tiny \decay{\Lb}{\Lz\mumu}};
 \end{scope}
\end{tikzpicture}
\begin{tikzpicture}
 \centering
 \node[anchor=south west,inner sep=0](image) at (0,0) {\includegraphics[height=0.33\textheight]{figs/BFs/kmumu_BF.pdf}};
 \begin{scope}[x={(image.south east)},y={(image.north west)}]
  %\draw[help lines,xstep=.1,ystep=.1] (0,0) grid (1,1);
  %\node[draw=none] at (0.45,0.85) {\tiny \BsTophimm};
 \end{scope}
\end{tikzpicture}
\begin{tikzpicture}
 \centering
 \node[anchor=south west,inner sep=0](image) at (0,0) {\includegraphics[height=0.33\textheight]{figs/BFs/ksmumu_BF.pdf}};
 \begin{scope}[x={(image.south east)},y={(image.north west)}]
  %\draw[help lines,xstep=.1,ystep=.1] (0,0) grid (1,1);
  %\node[draw=none] at (0.45,0.85) {\tiny \BsTophimm};
 \end{scope}
\end{tikzpicture}
\begin{tikzpicture}
 \centering
 \node[anchor=south west,inner sep=0](image) at (0,0) {\includegraphics[height=0.33\textheight]{figs/BFs/bukst_BF.pdf}};
 \begin{scope}[x={(image.south east)},y={(image.north west)}]
  %\draw[help lines,xstep=.1,ystep=.1] (0,0) grid (1,1);
  %\node[draw=none] at (0.45,0.85) {\tiny \BsTophimm};
 \end{scope}
\end{tikzpicture}
\end{center}

\end{frame}

%%%%%%%%%%%%%%%%%%%%%%%%%%%%%%%%%%%%%%%%%%%%%%%%%%%%%%%%%%%%%%%%
%Slide
%%%%%%%%%%%%%%%%%%%%%%%%%%%%%%%%%%%%%%%%%%%%%%%%%%%%%%%%%%%%%%%%%
\begin{frame}{Previous measurements: Angular analysis of $\decay{\Bd}{\KstP\mup\mun}$}

\begin{itemize}
\item \KstP is a P-wave (spin-1) state which decays to $\Kp\pim$
\item Decay fully described by $\qsq\equiv m_{\mu\mu}^{2}$ and three angles: $\thetal,\thetak,\phi$
\item Expression for angular distribution depends on the spin of the \Kstar meson
\end{itemize}

  \begin{columns}
   \begin{column}{0.6\textwidth}
  \scalebox{0.8}{\parbox{.5\linewidth}{%
  \begin{align*}
   \frac{1}{\deriv(\Gamma+\bar{\Gamma})/\deriv q^2} & \left. \frac{\deriv^3(\Gamma+\bar{\Gamma})}{\deriv\Omega}\right|_{\rm P-wave} = \\
    \frac{9}{32\pi} \Big[& 
     \tfrac{3}{4} (1-\textcolor{cobalt}{{F_{\rm L}}})\sin^2\thetak + \textcolor{cobalt}{{F_{\rm L}}}\cos^2\thetak\\
     &+ \tfrac{1}{4}(1-\textcolor{cobalt}{{F_{\rm L}}})\sin^2\thetak\cos 2\thetal \\
     &- \textcolor{cobalt}{{F_{\rm L}}} \cos^2\thetak\cos 2\thetal + \textcolor{cobalt}{{S_3}}\sin^2\thetak \sin^2\thetal \cos 2\phi\nonumber\\
     &+ \textcolor{cobalt}{{S_4}} \sin 2\thetak \sin 2\thetal \cos\phi + \textcolor{cobalt}{{S_5}}\sin 2\thetak \sin \thetal \cos \phi\nonumber\\
     &+ \tfrac{4}{3} \textcolor{cobalt}{{A_{\rm FB}}} \sin^2\thetak \cos\thetal + \textcolor{cobalt}{{S_7}} \sin 2\thetak \sin\thetal \sin\phi\nonumber\\
     &+ \textcolor{cobalt}{{S_8}} \sin 2\thetak \sin 2\thetal \sin\phi + \textcolor{cobalt}{{S_9}}\sin^2\thetak \sin^2\thetal \sin 2\phi  \Big]
  \end{align*}
  }}
   \end{column}
   \begin{column}{0.4\textwidth}
    \resizebox{\columnwidth}{!}{
    \begin{tikzpicture}
    %\draw[step=1cm,gray,very thin] (0,0) grid (7,4);
    %planes
    \draw[thick,gray] (0.2,3.5)--(2,3.5)--(5,0.5)--(3.2,0.5)--cycle;
    \draw[thick,gray] (3.,3.75)--(4.,0.25)--(6.25,0.25)--(5.25,3.75)--cycle;
    %axis
    \draw[ultra thick,->] (0.2,2.0) -- (6.8,2.0);
    %B
    \draw[fill] (3.5,2) circle (0.1);
    \node[draw=none] at (3.7,2.3){\tiny \Bz};
    %mumu
    \draw[fill,red] (2.5,2) circle (0.05);
    \draw[thick,->,red,>=stealth] (2.5,2) -- (1.5,3);
    \draw[thick,->,red,>=stealth] (2.5,2) -- (3.5,1);
    \node[draw=none,red] at (1.5,2.65){\tiny \mup};
    \node[draw=none,red] at (3.1,1.1){\tiny \mun};
    \draw[thick,red,->,>=stealth] ([shift=(0:0.45)]2.5,2) arc (0:132:0.45);
    \node[draw=none,red] at (2.6,2.2){\tiny $\theta_{\ell}$};

    %kpi
    \draw[fill,blue] (4.75,2) circle (0.05);
    \draw[thick,->,blue,>=stealth] (4.75,2) -- (4.375,3.25);
    \draw[thick,->,blue,>=stealth] (4.75,2) -- (5.125,0.75);
    \node[draw=none,blue] at (4.85,2.95){\tiny \Kp};
    \node[draw=none,blue] at (5.30,1.3){\tiny \pim};
    \draw[thick,blue,->,>=stealth] ([shift=(0:0.45)]4.75,2) arc (0:105:0.45);
    \node[draw=none,blue] at (4.92,2.22){\tiny $\theta_{K}$};

    \draw[thick,->,>=stealth] ([shift=(115:1.)]3.5,2.5) arc (115:155:1.);
    \node[draw=none] at (2.9,3.1){\tiny $\phi$};
    \end{tikzpicture}
    }
   \end{column}
  \end{columns}

  \begin{itemize}
  \item $\textcolor{cobalt}{{F_{\rm L}}}$, $\textcolor{cobalt}{{A_{\rm FB}}}$, $\textcolor{cobalt}{{S_i}}$ are the angular observables which are measured
  \begin{itemize}
    \item[\ding{70}] Can be compared with SM predictions
    \end{itemize}
  \item Often useful to measure ratios of observables
    \begin{itemize}
    \item[\ding{70}] e.g. $P'_{4,5} = \textcolor{cobalt}{S_{4,5}}/\sqrt{ \textcolor{cobalt}{F_{\rm L}}(1 -  \textcolor{cobalt}{F_{\rm L}})}$
    \end{itemize}
  \end{itemize}
\end{frame}

%%%%%%%%%%%%%%%%%%%%%%%%%%%%%%%%%%%%%%%%%%%%%%%%%%%%%%%%%%%%%%%%
%Slide
%%%%%%%%%%%%%%%%%%%%%%%%%%%%%%%%%%%%%%%%%%%%%%%%%%%%%%%%%%%%%%%%%
\begin{frame}{Previous measurements: Angular analysis of \BdToKstmmP}

\begin{itemize}
    \item[\ding{80}] Deviation in $P'_{5}$ at level of 2.9$\sigma$ in both bins [4.0,\,6.0] and [6.0,\,8.0] \gevgevcccc
\end{itemize}

\begin{center}
\begin{tikzpicture}
    \centering
    \node[anchor=south west,inner sep=0](image) at (0,0) {\includegraphics[width=0.5\textwidth]{figs/kpimm/introduction/P5prime.pdf}};
    \begin{scope}[x={(image.south east)},y={(image.north west)}]
      %\draw[help lines,xstep=.1,ystep=.1] (0,0) grid (1,1);
      \node[draw,ellipse,darkpastelred,minimum width=40pt,minimum height=50pt,thick,dashed] at (0.41,0.35) {};
      %\node[draw=none] at (0.4,0.875) {\tiny [JHEP 02 (2016) 104]};
    \end{scope}
  \end{tikzpicture}
\end{center}

\begin{itemize}
  \item[\ding{80}] Results form a {\bf consistent pattern} with the discrepancies in $\deriv\BF/\deriv\qsq$ measurements
  \item[\ding{80}] Two possible interpretations that can explain both discrepancies simultaneously
  \begin{itemize}
    \item[\ding{182}] Contribution of NP, e.g. a new vector Z' boson
    \item[\ding{183}] Underestimation of the QCD (charm loop) effects 
  \end{itemize}
\end{itemize}
\end{frame}

%%%%%%%%%%%%%%%%%%%%%%%%%%%%%%%%%%%%%%%%%%%%%%%%%%%%%%%%%%%%%%%%%
%Slide
%%%%%%%%%%%%%%%%%%%%%%%%%%%%%%%%%%%%%%%%%%%%%%%%%%%%%%%%%%%%%%%%%
\begin{frame}{\BdToKpimm in the \Kstarfourteenthirty region \hspace{0pt plus 1 filll} {\small \bf \textcolor{black}{[JHEP 12 (2016) 065]}}}

\begin{itemize}
\item Analyses of \BdToKpimm at \lhcb have focused on the (P-wave) \KstarP
\item Region of \mkpi $\sim$1430\mevcc contains contributions from S-, P- and D-wave $K^\ast$ states (spin-0,1,2)
\begin{itemize}
  \item Leads to a very complicated angular expression!
\end{itemize}
\end{itemize}

\begin{center}
\begin{tikzpicture}
 \node[anchor=south west,inner sep=0](image) at (0,0) {\includegraphics[width=0.5\textwidth]{figs/kpimm/introduction/full-mkpi.pdf}};
 \begin{scope}[x={(image.south east)},y={(image.north west)}]
  %\draw[help lines,xstep=.1,ystep=.1] (0,0) grid (1,1);
  \node[draw=none] at (0.25,0.75) {\footnotesize \color{red}{\KstarP}};
  \draw[->,thick,red] (0.25, 0.7) -- (0.35,0.4);
  \node[draw=none] at (0.75,0.6) {\footnotesize \color{blue}{New region}};
  \draw[->,thick,blue] (0.75, 0.55) -- (0.78,0.37);
 \end{scope}
\end{tikzpicture}
\end{center}

\begin{itemize}
  \item Will present the first measurements of the differential branching fraction and angular analysis in this new \mkpi region
\end{itemize}

\end{frame}

%%%%%%%%%%%%%%%%%%%%%%%%%%%%%%%%%%%%%%%%%%%%%%%%%%%%%%%%%%%%%%%%%
%Slide
%%%%%%%%%%%%%%%%%%%%%%%%%%%%%%%%%%%%%%%%%%%%%%%%%%%%%%%%%%%%%%%%%
\begin{frame}{\mkpimm invariant mass distribution}

\begin{itemize}
  \item Fit to invariant mass distribution required to separate signal from background contribution
  \item Signal modelled with sum of two Gaussians with power-law tail on the low-mass side
  \item Combinatorial background is modelled using an exponential function
  \item $\sim230$ signal candidates in $1.1<\qsq<6.0\gevgevcccc$
\end{itemize}

\begin{center}
\includegraphics[width=0.6\textwidth]{figs/kpimm/massfit/fitKpimumu_q2_1p1_6p0.pdf}
\end{center}

\end{frame}

%%%%%%%%%%%%%%%%%%%%%%%%%%%%%%%%%%%%%%%%%%%%%%%%%%%%%%%%%%%%%%%%%
%Slide
%%%%%%%%%%%%%%%%%%%%%%%%%%%%%%%%%%%%%%%%%%%%%%%%%%%%%%%%%%%%%%%%%
\begin{frame}{Efficiency studies}

\begin{itemize}
  \item The efficiency of the signal candidate selection varies over the distributions of the decay angles $\thetal,\thetak,\phi$
  \item In order to perform an angular analysis, need to carefully understand this effect
  \item In this analysis the efficiency shape is modelled using a Legendre polynomial paramerisation
  \begin{itemize}
    \item Requires 1764 legendre coefficients!
  \end{itemize}
\end{itemize}

\bigskip

\begin{center}
\includegraphics[width=0.32\textwidth]{figs/kpimm/acceptance/Fig3a.pdf}
\includegraphics[width=0.32\textwidth]{figs/kpimm/acceptance/Fig3b.pdf}
\includegraphics[width=0.32\textwidth]{figs/kpimm/acceptance/Fig3c.pdf}
\end{center}

\end{frame}

%%%%%%%%%%%%%%%%%%%%%%%%%%%%%%%%%%%%%%%%%%%%%%%%%%%%%%%%%%%%%%%%%
%Slide
%%%%%%%%%%%%%%%%%%%%%%%%%%%%%%%%%%%%%%%%%%%%%%%%%%%%%%%%%%%%%%%%%
\begin{frame}{Differential branching fraction ($\deriv\BF/\deriv\qsq$)}

\begin{itemize}
\item First $\deriv\BF/\deriv\qsq$ measurement of \BdToKpimm in this region of \mkpi 
\end{itemize}

\begin{center}
\includegraphics[width=0.6\textwidth]{figs/kpimm/bf/dbfdq2.pdf}
\end{center}

\begin{itemize}
\item Currently no SM predictions
\begin{itemize}
\item[\ding{80}] Hope this analysis stimulates further theory effort in this area
\end{itemize}
\end{itemize}

\end{frame}

%%%%%%%%%%%%%%%%%%%%%%%%%%%%%%%%%%%%%%%%%%%%%%%%%%%%%%%%%%%%%%%%%
%Slide
%%%%%%%%%%%%%%%%%%%%%%%%%%%%%%%%%%%%%%%%%%%%%%%%%%%%%%%%%%%%%%%%%
\begin{frame}{Angular moments analysis}

\begin{itemize}
\item Angular distribution for \BdToKpimm decays for decays encompassing S-, P- and D-wave \Kstar states can be expressed in an orthonormal basis of angular functions, $f_i(\thetal,\thetak,\phi)$,
\begin{equation*}
\frac{\deriv\Gamma }{\deriv\qsq\,\deriv\Omega} \propto \sum^{41}_{i=1} f_i (\thetal,\thetak,\phi) \Gamma_i(\qsq)
%\quad\mbox{with}\quad
%\int f_if_j\deriv\costhetal\deriv\costhetak\deriv\phi = \delta_{ij}
\end{equation*}
\item 41 angular terms compared to 11 for \BdToKstmmP!
\item Define 40 normalised angular moments which forms set of observables
\end{itemize}
\begin{equation*}
\overline{\Gamma}_i(\qsq) = \frac{\Gamma_{i}(\qsq)}{\Gamma_{1}(\qsq)}
\end{equation*}

\end{frame}

%%%%%%%%%%%%%%%%%%%%%%%%%%%%%%%%%%%%%%%%%%%%%%%%%%%%%%%%%%%%%%%%%
%Slide
%%%%%%%%%%%%%%%%%%%%%%%%%%%%%%%%%%%%%%%%%%%%%%%%%%%%%%%%%%%%%%%%%
\begin{frame}{Angular moments analysis}

\begin{itemize}
  \item Observables measured using an angular moments analysis
  \begin{itemize}
    \item[\ding{70}] Likelihood fit impossible due to complicated angular expression and limited statistics
  \end{itemize}
\end{itemize}

\begin{itemize}
\item Calculate background-subtracted and acceptance-corrected moments
\end{itemize}

\begin{columns}
\begin{column}{0.6\textwidth}
\begin{equation*}
  \begin{aligned}
  \Gamma_i =&  \sum_{k=1}^{n_{\rm sig}} w_{k}f_i(\Omega_k)  - x\sum_{k=1}^{n_{\rm bkg}} w_{k}f_i(\Omega_k)\\
  C_{ij} =& \sum_{k=1}^{n_{\rm sig}} w^{2}_{k}f_i(\Omega_k)f_j(\Omega_k)   + x^2\sum_{k=1}^{n_{\rm bkg}} w^{2}_{k}f_i(\Omega_k)f_j(\Omega_k)
  \end{aligned}
  \label{eqn:mom:sigbkgacc}
\end{equation*}
\end{column}
\begin{column}{0.4\textwidth}
\begin{centering}
\includegraphics[width=\textwidth]{figs/moments-regions.pdf}
\end{centering}
\end{column}
\end{columns}

\bigskip

\begin{itemize}
\item $x$ is the ratio $\tilde{n}^{\rm bkg}_{\rm sig}/n_{\rm bkg}$ used to perform the background subtraction
\item $w_{k}$ is the reciprocal of the efficiency, $1/\varepsilon_{k}$
\end{itemize}

\end{frame}

%%%%%%%%%%%%%%%%%%%%%%%%%%%%%%%%%%%%%%%%%%%%%%%%%%%%%%%%%%%%%%%%%
%Slide
%%%%%%%%%%%%%%%%%%%%%%%%%%%%%%%%%%%%%%%%%%%%%%%%%%%%%%%%%%%%%%%%%
\begin{frame}{Angular moments analysis}
\begin{columns}
\begin{column}{0.55\textwidth}
\begin{itemize}
  \item Currently no SM predictions for the moments 
  \begin{itemize}
    \item[\ding{80}] Hope measurement will stimulate futher theory effort
  \end{itemize}
\end{itemize}
\begin{itemize}
  \item Specific moments point towards large interference between S- and P-wave states 
  \item Can also estimate D-wave fraction
  \begin{itemize}
    \item $F_{\rm D} <$ 0.29 @ 95\% C.L.
    \item Low w.r.t expectation from previous measurement of \decay{\Bd}{\jpsi(\to\mumu) \Kp\pim}
\end{itemize}
\end{itemize}
\end{column}
\begin{column}{0.45\textwidth}
\centering
\includegraphics[height=0.44\textheight]{figs/kpimm/angular-analysis/mom_results_2_21.pdf}\\
\includegraphics[height=0.44\textheight]{figs/kpimm/angular-analysis/mom_results_22_41.pdf}
\end{column}
\end{columns}
\end{frame}

%%%%%%%%%%%%%%%%%%%%%%%%%%%%%%%%%%%%%%%%%%%%%%%%%%%%%%%%%%%%%%%%%
%Slide
%%%%%%%%%%%%%%%%%%%%%%%%%%%%%%%%%%%%%%%%%%%%%%%%%%%%%%%%%%%%%%%%%
\begin{frame}{Summary \& Outlook}

\begin{overlayarea}{\textwidth}{0.9\textheight}

\bigskip

\begin{itemize}
  \item[\ding{72}] Novel idea to use upstream tracking to speed up the Upgrade trigger a success!
  \item[\ding{72}] New algorithm adopted into the default tracking sequence
\end{itemize}
  \begin{itemize}
  \item[\ding{80}] First measurements of the differential branching fraction and angular moments of \BdToKpimm in \Kstarfourteenthirty region
  \item[\ding{80}] With additional input from theory, results could provide further complementary measurements of \btosll transitions
  \begin{itemize}
    \item[\ding{70}] Improve understanding of the observed pattern of deviations with respect to SM predictions
  \end{itemize}
\end{itemize}

\bigskip

\only<2>{
\begin{fancyquotes}
...when you have eliminated all the Standard Model explanations, whatever remains, however improbable, must be New Physics.\par\emph{Dr Joaquim Matias}
\end{fancyquotes}
}
\end{overlayarea}
\end{frame}


%%%%%%%%%%%%%%%%%%%%%%%%%%%%%%%%%%%%%%%%%%%%%%%%%%%%%%%%%%%%%%%%%
%Backup
%%%%%%%%%%%%%%%%%%%%%%%%%%%%%%%%%%%%%%%%%%%%%%%%%%%%%%%%%%%%%%%%%
\appendix

%%%%%%%%%%%%%%%%%%%%%%%%%%%%%%%%%%%%%%%%%%%%%%%%%%%%%%%%%%%%%%%%%
%Slide
%%%%%%%%%%%%%%%%%%%%%%%%%%%%%%%%%%%%%%%%%%%%%%%%%%%%%%%%%%%%%%%%%
{\usebackgroundtemplate{\includegraphics[width=1.05\paperwidth]{figs/cavern.jpg}}
 
 \begin{frame}[plain]
 \vspace{8.75cm}
 \hspace{-0.75cm}
 \Huge\color{anti-flashwhite}{Backup}
 \end{frame}
}

% %%%%%%%%%%%%%%%%%%%%%%%%%%%%%%%%%%%%%%%%%%%%%%%%%%%%%%%%%%%%%%%%%
% %Slide
% %%%%%%%%%%%%%%%%%%%%%%%%%%%%%%%%%%%%%%%%%%%%%%%%%%%%%%%%%%%%%%%%%
% \begin{frame}\frametitle{\bquark\bquarkbar pair production in \proton\proton collisions}

% \begin{center}
% \begin{tikzpicture}
%  \centering
%  \node[anchor=south west,inner sep=0](image) at (0,0) {\includegraphics[width=0.4\paperwidth]{figs/detector/gluon_fusion.eps}};
%  \begin{scope}[x={(image.south east)},y={(image.north west)}]
%   %\draw[help lines,xstep=.1,ystep=.1] (0,0) grid (1,1);
%   \node[draw=none] at (0.5,-0.2) {gluon-gluon fusion};
%  \end{scope}
% \end{tikzpicture}
% \begin{tikzpicture}
%  \centering
%  \node[anchor=south west,inner sep=0](image) at (0,0) {\includegraphics[width=0.4\paperwidth]{figs/detector/quark_antiquark_annihilation.eps}};
%  \begin{scope}[x={(image.south east)},y={(image.north west)}]
%   %\draw[help lines,xstep=.1,ystep=.1] (0,0) grid (1,1);
%   \node[draw=none] at (0.5,-0.2) {quark-antiquark annihilation};
%  \end{scope}
% \end{tikzpicture}
% \end{center}

% \end{frame}

%%%%%%%%%%%%%%%%%%%%%%%%%%%%%%%%%%%%%%%%%%%%%%%%%%%%%%%%%%%%%%%%%
%Slide
%%%%%%%%%%%%%%%%%%%%%%%%%%%%%%%%%%%%%%%%%%%%%%%%%%%%%%%%%%%%%%%%%
% \begin{frame}\frametitle{}

%   \begin{center}
%   \resizebox{\columnwidth}{!}{
%     \begin{tikzpicture}
%       %\draw[step=1cm,gray,very thin] (0,0) grid (15,4);
      
%       \node[draw=none] at (2,3.5) {Beauty hadrons};
        
%       \node[draw=none,blue] at (3,1.75) {L};
%       \draw[ultra thick,dashed,blue] (2,1) -- (4,2);
%       \draw[thick] (4,2) -- (5,3.8);
%       %\node[draw=none,darkpastelgreen] at (5.3,4.05) {$\mu^{+}$};
%       \draw[thick] (4,2) -- (5.5,3.5);
%       %\node[draw=none,darkpastelgreen] at (5.75,3.75) {$\mu^{-}$};
%       \draw[thick] (4,2) -- (6,3.2);
%       \draw[thick] (4,2) -- (6.2,3.);
%       \draw[thick] (4,2) -- (6.2,2.6);
      
%       \node[draw=none] at (4.2,1.8) {SV};
        
%       \draw[ultra thick,densely dotted,darkorchid] (4,2) -- (2,0.2);
%       \draw[ultra thick,densely dotted,darkorchid] (2,1) -- (2.35,0.5);
%       \node[draw=none,darkorchid] at (2,0.5) {IP};
      
%       \node[draw=none,darkpastelred] at (2,1.6) {PV};
%       \node[draw, fill, orange, star, star points=11,scale=1.0] at (2,1){};
%       \node[draw, fill, darkpastelred, star, star points=11,scale=0.5] at (2,1){};
      
%       \draw[ultra thick,->] (0,1) -- (1.8,1);
%       \draw[ultra thick,->] (7,1) -- (2.2,1);
%       \node[draw=none] at (0.2,0.8) {\proton};
%       \node[draw=none] at (6.8,0.8) {\proton};
      
%       \node[draw=none] at (10,3.5) {Charm hadrons};
      
%       \node[draw=none,blue] at (10.5,1.5) {L};
%       \draw[ultra thick,dashed,blue] (10,1) -- (11,1.5);
%       \draw[thick] (11,1.5) -- (12.5,3.8);
%       \draw[thick] (11,1.5) -- (14,3);
        
%       \node[draw=none] at (11.3,1.35) {SV};
        
%       \draw[ultra thick,densely dotted,darkorchid] (11,1.5) -- (10,0.4);
%       \draw[ultra thick,densely dotted,darkorchid] (10,1) -- (10.3,0.7);
%       \node[draw=none,darkorchid] at (10.5,0.5) {IP};
      
%       \node[draw=none,darkpastelred] at (10,1.6) {PV};
%       \node[draw, fill, orange, star, star points=11,scale=1.0] at (10,1){};
%       \node[draw, fill, darkpastelred, star, star points=11,scale=0.5] at (10,1){};
      
%       \draw[ultra thick,->] (8,1) -- (9.8,1);
%       \draw[ultra thick,->] (15,1) -- (10.2,1);
%       \node[draw=none] at (8.2,0.8) {\proton};
%       \node[draw=none] at (14.8,0.8) {\proton};
      
%     \end{tikzpicture}
%     }
%     \bigskip
    
%     \begin{columns}
%       \begin{column}{0.49\textwidth}
%         \begin{myenv}{Beauty signatures}[linecolor=barcolor]
%           \begin{itemize}
%           \item \Bz mass 5.28 GeV
%           \begin{itemize}
%           \item[\ding{70}] daughter \pt \orderof(1 GeV)
%           \end{itemize}
%           \item $\tau \sim 1.6\ps$
%           \begin{itemize}
%           \item[\ding{70}] L $\sim 1\cm$ $\to$ large IP
%           \end{itemize}
%           %\item Common signature: displaced muon pair
%           \end{itemize}
%         \end{myenv}
%       \end{column}
%       \begin{column}{0.49\textwidth}
%         \begin{myenv}{Charm signatures}[linecolor=barcolor]
%         \begin{itemize}
%           \item \Dz mass 1.86 GeV
%           \begin{itemize}
%           \item[\ding{70}] large daughter \pt
%           \end{itemize}
%           \item $\tau \sim 0.4\ps$
%           \begin{itemize}
%           \item[\ding{70}] L $\sim 0.4 \cm$ $\to$ smaller IP
%           \end{itemize}
%           %\item Common signature: displaced muon pair
%           \end{itemize}
%         \end{myenv}
%       \end{column}
%     \end{columns}
    
%   \end{center}
% \end{frame}

% %%%%%%%%%%%%%%%%%%%%%%%%%%%%%%%%%%%%%%%%%%%%%%%%%%%%%%%%%%%%%%%%%
% %Slide
% %%%%%%%%%%%%%%%%%%%%%%%%%%%%%%%%%%%%%%%%%%%%%%%%%%%%%%%%%%%%%%%%%
\begin{frame}\frametitle{Tracking definitions}

\begin{equation*}
\text{Reconstruction efficiency} = \frac{N_{\text{reconstructible~and~reconstructed}}}{N_{\text{reconstructible}}}
\end{equation*}

\bigskip

\begin{columns}
\begin{column}{0.2\textwidth}
\end{column}
\begin{column}{0.6\textwidth}
\begin{myenv}{Reconstructible requirements}[linecolor=barcolor]
\centering
  \begin{tabular}{c|c}
    Common req. & Upstream req.\\
    \hline
    !electron & Reco. in VELO\\
    $2<\eta<5$ & 3+ TT layers fired\\
    $\pt>0.5\gevc$ \\
    \bquark-hadron daughter \\
    \velo + T hits \\
  \end{tabular}
\end{myenv}
\end{column}
\begin{column}{0.2\textwidth}
\end{column}
\end{columns}
\end{frame}

% %%%%%%%%%%%%%%%%%%%%%%%%%%%%%%%%%%%%%%%%%%%%%%%%%%%%%%%%%%%%%%%%%
% %Slide
% %%%%%%%%%%%%%%%%%%%%%%%%%%%%%%%%%%%%%%%%%%%%%%%%%%%%%%%%%%%%%%%%%
\begin{frame}\frametitle{Tracking definitions}

\begin{equation*}
\text{Ghost rate}  = \frac{N_{\text{ghost~tracks}}}{N_{\text{tracks}}}
\end{equation*}

\bigskip

\begin{columns}
\begin{column}{0.25\textwidth}
\end{column}
\begin{column}{0.5\textwidth}
\begin{myenv}{Track requirements}[linecolor=barcolor]
\centering
\begin{tabular}{c}
  $2<\eta<5$ \\
  $\pt>0.5\gevc$ \\
\end{tabular}
\medskip
\end{myenv}
\end{column}
\begin{column}{0.25\textwidth}
\end{column}
\end{columns}
\end{frame}

% %%%%%%%%%%%%%%%%%%%%%%%%%%%%%%%%%%%%%%%%%%%%%%%%%%%%%%%%%%%%%%%%%
% %Slide
% %%%%%%%%%%%%%%%%%%%%%%%%%%%%%%%%%%%%%%%%%%%%%%%%%%%%%%%%%%%%%%%%%
\begin{frame}\frametitle{VeloUT: Wrong side tracks}

\begin{itemize}
  \item Fraction of truth matched tracks with at least one hit on the opposite side of the extrapolated \velo track in the $x$-$z$ plane
\end{itemize}

\begin{center}
\includegraphics[width=0.4\textwidth]{figs/upstream-tracking-upgrade/wrong_side_hits.pdf}
\end{center}
\end{frame}

%%%%%%%%%%%%%%%%%%%%%%%%%%%%%%%%%%%%%%%%%%%%%%%%%%%%%%%%%%%%%%%%
%% Slide
%%%%%%%%%%%%%%%%%%%%%%%%%%%%%%%%%%%%%%%%%%%%%%%%%%%%%%%%%%%%%%%%
\begin{frame}{Upstream tracking for \lhcb Run 2 (2015-2018)}

\begin{itemize}
  \item[$\blacktriangleright$] Following improved performance achieved using upstream tracks, similar strategy developed for Run 2
  \item[$\blacktriangleright$]   New VeloTT algorithm created based on the optimised VeloUT algorithm
  \begin{itemize}
    \item[\ding{80}] Similar improvements achieved
    \item[\ding{80}] Allows IP requirements to be removed
    \item[\ding{80}] Adopted into the default tracking sequence for the first stage of the HLT
  \end{itemize}
\end{itemize}

\begin{itemize}
  \item[\ding{80}] Greatly improved signal efficiency for charm physics
  \item[\ding{80}] Lifetime unbiased triggering on hadronic final states for first time
\end{itemize}

\begin{itemize}
\item[\ding{80}] First HEP experiment to implement fully automatic tracking system alignment, PID calibration and track reconstruction online
\end{itemize}

\end{frame}

%%%%%%%%%%%%%%%%%%%%%%%%%%%%%%%%%%%%%%%%%%%%%%%%%%%%%%%%%%%%%%%%%
%Slide
%%%%%%%%%%%%%%%%%%%%%%%%%%%%%%%%%%%%%%%%%%%%%%%%%%%%%%%%%%%%%%%%%
\begin{frame}{VeloTT: Optimised peformance}

\begin{columns}
\begin{column}{0.65\textwidth}
\begin{itemize}
  \item Large improvement in the efficiency ($+5$\%)
  \begin{itemize}
    \item[\ding{70}] Now flat in \ptot
  \end{itemize}
  \item Huge reduction in the execution time ($\times$65!)
  \item Slight increase in the ghost rate ($+4$\%)
  \begin{itemize}
    \item[\ding{70}] Can be reduced in offline analysis
  \end{itemize}
\end{itemize}

\bigskip

\begin{mdframed}[linecolor=barcolor]
\begin{center}
\resizebox{\columnwidth}{!}{
\begin{tabular}{c|c|c|c}
  \velott & Efficiency [\%] & Ghost rate [\%] & Timing [ms] \\ 
  \hline
  Run I  &  92.74  &  7.21  &  32.50  \\
  Run II  &  97.77  &  11.60  &  \hphantom{0}0.50   \\
 \end{tabular}
 }
\end{center}
\end{mdframed}
\end{column}

\begin{column}{0.35\textwidth}
\centering
\begin{figure}
\vspace*{-1cm}
\includegraphics[height=0.475\textheight]{figs/upstream-tracking-run2/VeloTT-eff-p.pdf}\\
\includegraphics[height=0.475\textheight]{figs/upstream-tracking-run2/VeloTT-gr-p.pdf}
\end{figure}
\end{column}
\end{columns}

\end{frame}

%%%%%%%%%%%%%%%%%%%%%%%%%%%%%%%%%%%%%%%%%%%%%%%%%%%%%%%%%%%%%%%%%
%Slide
%%%%%%%%%%%%%%%%%%%%%%%%%%%%%%%%%%%%%%%%%%%%%%%%%%%%%%%%%%%%%%%%%
\begin{frame}{VeloTT-Forward: Optimised peformance}

\begin{columns}
\begin{column}{0.65\textwidth}
\begin{itemize}
  \item Significant reduction in the execution time ($\times$3)
  \item Large reduction in the ghost rate ($\times$3)
  \item Some loss of efficiency ($-4$\%)
\end{itemize}

\bigskip

\begin{mdframed}[linecolor=barcolor]
\begin{center}
\resizebox{\columnwidth}{!}{
\begin{tabular}{c|c|c|c}
    & Efficiency [\%] & Ghost rate [\%] & Timing [ms] \\
   \hline
  Velo-Forward  & 93.15  & 46.86  &  13.71 \\
  VeloTT-Forward  & 89.23  & 17.13  &  0.50+4.08 \\
 \end{tabular}
 }
\end{center}
\end{mdframed}
\end{column}

\begin{column}{0.35\textwidth}
\centering
\begin{figure}
\vspace*{-1cm}
\includegraphics[height=0.475\textheight]{figs/upstream-tracking-run2/Forward-eff-p.pdf}\\
\includegraphics[height=0.475\textheight]{figs/upstream-tracking-run2/Forward-gr-p.pdf}
\end{figure}
\end{column}
\end{columns}

\end{frame}

%%%%%%%%%%%%%%%%%%%%%%%%%%%%%%%%%%%%%%%%%%%%%%%%%%%%%%%%%%%%%%%%%
%Slide
%%%%%%%%%%%%%%%%%%%%%%%%%%%%%%%%%%%%%%%%%%%%%%%%%%%%%%%%%%%%%%%%%
\begin{frame}{Theoretical formalism}

\begin{itemize}
  \item Rare \bquark-hadron decays are a multi-scale problem: $m_{W} \gg m_{\bquark} > \Lambda_{\rm QCD}$
  \item Measurements interpreted in Operator Product Expansion framework
  \begin{itemize}
  \item[\ding{70}] All degrees of freedom above a given energy scale are integrated out
  \item[\ding{70}] Introduce set of Wilson coefficients, $\mathcal{C}_{i}$, and local operators, $\mathcal{O}_{i}$, encoding coupling strength and Lorentz structure
  \end{itemize}
\end{itemize}
\begin{equation*}
\mathcal{H}_{\rm eff} = - \frac{4G_{F}}{\sqrt{2}}V_{tb}V^{*}_{ts}\sum_{i}(\mathcal{C}_{i}^{\rm SM}+\mathcal{C}_{i}^{\rm NP})\mathcal{O}_{i}
\end{equation*}
  
\begin{itemize}
\item \btosll transitions sensitive to $\mathcal{C}_{7}$, $\mathcal{C}_{9}$, $\mathcal{C}_{10}$
\end{itemize}

\medskip

\centering
\begin{tikzpicture}
\node[anchor=south west,inner sep=0](image) at (0,0) {\includegraphics[width=0.3\textwidth]{figs/theory/btosll_penguin.eps}};
\begin{scope}[x={(image.south east)},y={(image.north west)}]
%\draw[help lines,xstep=.1,ystep=.1] (0,0) grid (1,1);
\node[draw=none] at (0.53,0.96) {\scriptsize \Wm};
\node[draw=none] at (0.3,0.6) {\scriptsize \tquark};
\node[draw=none] at (0.56,0.27) {\scriptsize {\small\color{bleudefrance}{\Pgamma}}\color{black}{,} \color{bostonuniversityred}{$\Z^{0}$}};
\end{scope}
\end{tikzpicture}
\begin{tikzpicture}
\node[anchor=south west,inner sep=0](image) at (0,0) {\includegraphics[width=0.3\textwidth]{figs/theory/btosll_effective.eps}};
\begin{scope}[x={(image.south east)},y={(image.north west)}]
%\draw[help lines,xstep=.1,ystep=.1] (0,0) grid (1,1);
\node[draw=none] at (0.53,0.58) {\scriptsize \color{bleudefrance}{$\mathcal{C}_{7}$}\color{black}{,} \color{bostonuniversityred}{$\mathcal{C}_{9,10}$}};
\end{scope}
\end{tikzpicture}
\end{frame}

%%%%%%%%%%%%%%%%%%%%%%%%%%%%%%%%%%%%%%%%%%%%%%%%%%%%%%%%%%%%%%%%%
%Slide
%%%%%%%%%%%%%%%%%%%%%%%%%%%%%%%%%%%%%%%%%%%%%%%%%%%%%%%%%%%%%%%%%
\begin{frame}{Operators}

\begin{alignat*}{2}
&\order_{7} = \frac{e}{g^{2}}m_{\bquark}(\squarkbar\sigma_{\mu\nu}P_{R}\bquark)F^{\mu\nu}~~~~~~~&&\order_{7}^{\prime} = \frac{e}{g^{2}}m_{\bquark}(\squarkbar\sigma_{\mu\nu}P_{L}\bquark)F^{\mu\nu} \nonumber \\
&\order_{9} = \frac{e^{2}}{g^{2}}(\squarkbar\gamma_{\mu}P_{L}\bquark)(\ellbar\gamma^{\mu}\ell) &&\order_{9}^{\prime} = \frac{e^{2}}{g^{2}}(\squarkbar\gamma_{\mu}P_{R}\bquark)(\ellbar\gamma^{\mu}\ell) \nonumber \\
&\order_{10} = \frac{e^{2}}{g^{2}}(\squarkbar\gamma_{\mu}P_{L}\bquark)(\ellbar\gamma^{\mu}\gamma_{5}\ell) &&\order_{10}^{\prime} = \frac{e^{2}}{g^{2}}(\squarkbar\gamma_{\mu}P_{R}\bquark)(\ellbar\gamma^{\mu}\gamma_{5}\ell) \\
&\order_{S} = \frac{e^{2}}{16\pi^{2}}m_{\bquark}(\squarkbar P_{R}\bquark)(\ellbar\ell) &&\order_{S}^{\prime} = \frac{e^{2}}{16\pi^{2}}m_{\bquark}(\squarkbar P_{L}\bquark)(\ellbar\ell) \nonumber\\
&\order_{P} = \frac{e^{2}}{16\pi^{2}}m_{\bquark}(\squarkbar P_{R}\bquark)(\ellbar\gamma_{5}\ell) &&\order_{P}^{\prime} = \frac{e^{2}}{16\pi^{2}}m_{\bquark}(\squarkbar P_{L}\bquark)(\ellbar\gamma_{5}\ell) \nonumber
\end{alignat*}

\end{frame}

%%%%%%%%%%%%%%%%%%%%%%%%%%%%%%%%%%%%%%%%%%%%%%%%%%%%%%%%%%%%%%%%%
%Slide
%%%%%%%%%%%%%%%%%%%%%%%%%%%%%%%%%%%%%%%%%%%%%%%%%%%%%%%%%%%%%%%%%
\begin{frame}{Interpretation}
\begin{columns}
\begin{column}{0.5\textwidth}

\begin{tikzpicture}
 \centering
 \node[anchor=south west,inner sep=0](image) at (0,0) {\includegraphics[trim={2.5cm 20cm 10.7cm 2.4cm},clip,width=1.0\textwidth]{figs/kpimm/introduction/c9c10.pdf}};
 \begin{scope}[x={(image.south east)},y={(image.north west)}]
 %\draw[help lines,xstep=.1,ystep=.1] (0,0) grid (1,1);
  \node[draw=none] at (0.32,0.94) {\bf \scriptsize [arXiv:1503.06199]};
\end{scope}
\end{tikzpicture}
\centering

{\color{applegreen}{Branching fractions}}, {\color{antiquebrass}{angular observables}}, {\color{airforceblue}{combined}} 
\end{column}
\begin{column}{0.5\textwidth}
\begin{itemize}
\item Several attempts to interpret LHCb data by performing global fits
\item Consistent picture, data favours modified vector coupling ($\mathcal{C}_{9}^{\rm NP}$) at $\sim[3,4]\sigma$
\end{itemize}

\bigskip

\begin{myenv}{Possible interpretation}[linecolor=barcolor]
\begin{itemize}
\setlength{\itemindent}{0.5em}
\item NP physics scenario, e.g. new vector $Z^{'}$, leptoquarks, etc
\item Problem with our understanding of QCD, e.g. not properly estimating contribution from charm loops
\end{itemize}
\end{myenv}

\end{column}
\end{columns}
\end{frame}

%%%%%%%%%%%%%%%%%%%%%%%%%%%%%%%%%%%%%%%%%%%%%%%%%%%%%%%%%%%%%%%%%
%Slide
%%%%%%%%%%%%%%%%%%%%%%%%%%%%%%%%%%%%%%%%%%%%%%%%%%%%%%%%%%%%%%%%%
\begin{frame}{\BdToKpimm in the \Kstarfourteenthirty region \hspace{0pt plus 1 filll} {\small \bf \textcolor{black}{[JHEP 12 (2016) 065]}}}
\begin{itemize}
\item[$\blacktriangleright$] Measurements performed in $1330<\mkpi<1530\mevcc$ region at low \qsq
\item[]
\item[\ding{182}] Differential branching fraction as a function of \qsq
  \begin{itemize}
    \item[\ding{70}] 5 \qsq bins: [0.1,\,0.98], [1.1,\,2.5], [2.5,\,4.0], [4.0,\,6.0], [6.0,\,8.0]~\gevgevcccc
    \item[\ding{70}] Normalised to \BdToJPsiKstP
    \item[\ding{70}] Never previously measured
  \end{itemize}
\item[\ding{183}] Angular analysis
  \begin{itemize}
    \item[\ding{70}] Single \qsq bin: [1.1,\,6.0]~\gevgevcccc
    \item[\ding{70}] S-, P- and D-wave contributions considered for first time
    \item[\ding{70}] Requires new orthonormal basis of angular functions 
    % \item[\ding{70}] 40 normalised moments ($\overline{\Gamma}_{2}$-$\overline{\Gamma}_{41}$)
  \end{itemize}
  
\end{itemize}
\end{frame}

%%%%%%%%%%%%%%%%%%%%%%%%%%%%%%%%%%%%%%%%%%%%%%%%%%%%%%%%%%%%%%%%%
%Slide
%%%%%%%%%%%%%%%%%%%%%%%%%%%%%%%%%%%%%%%%%%%%%%%%%%%%%%%%%%%%%%%%%
\begin{frame}{Selection of \BdToKpimm signal candidates}

\vspace{-0.5cm}

\begin{itemize}
 \item[\ding{80}] Analysis performed using the full \lhcb Run I data sample (3\invfb)
\end{itemize}

\begin{itemize}
  \item[$\blacktriangleright$] Candidates are required to have `fired' trigger at each stage
  \item[$\blacktriangleright$] Cut based selection applied to select candidates of the form $\decay{\Bz}{X\mumu}$
  \item[$\blacktriangleright$] Multivariate classifier used to reduce combinatorial background
  \begin{itemize}
  \item[\ding{70}] Kinematic, isolation and PID variables used as input
  \end{itemize}
  \item[$\blacktriangleright$] Exclusive backgrounds removed by specific vetoes
  \begin{itemize}
  \item[\ding{70}] \BdToKpimm ($\kaon \leftrightarrow \pion$) 
  \item[\ding{70}] \BdToJPsiKpi ($\pion \leftrightarrow \muon$, $\kaon \leftrightarrow \muon$) 
  \item[\ding{70}] \BdToPsitwosKpi ($\pion \leftrightarrow \muon$, $\kaon \leftrightarrow \muon$) 
  \item[\ding{70}] \LbTopKmm ($\proton \to \pion$, $\proton \to \kaon~\&~\kaon \to \pion$) 
  \item[\ding{70}] \BuToKmm (random \pion) 
  \end{itemize}
\end{itemize}

\tikzoverlay at (9cm,3cm) {
\begin{tikzpicture}
\node[anchor=south west,inner sep=0](image) at (0,0) {\includegraphics[width=4.5cm]{figs/kpimm/selection/jpsi_pimu_fit.pdf}};
\node[draw=none] at (1.15,4.5) {\tiny $\BdToJPsiKpi mis$-$id$};
\end{tikzpicture} 
};

\end{frame}

%%%%%%%%%%%%%%%%%%%%%%%%%%%%%%%%%%%%%%%%%%%%%%%%%%%%%%%%%%%%%%%%%
%Slide
%%%%%%%%%%%%%%%%%%%%%%%%%%%%%%%%%%%%%%%%%%%%%%%%%%%%%%%%%%%%%%%%%
\begin{frame}{\BdToKpimm: Expected resonant contributions}

\begin{center}
\begin{tabular}{c|c|c|c|r}
Resonance & ${\rm J}^{P}$ & Mass [$\mathrm{Me\kern -0.1em V\!/}c^2$] & Full width [$\mathrm{Me\kern -0.1em V\!/}c^2$]  & $\mathcal{B}(K\pi)~[\%]$ \\
\hline
$K^\ast(1410)^0$ & $1^{-}$& $\hphantom{0.}1414 \pm 15\hphantom{.}$& $232 \pm 21\hphantom{0}$  & $6.6 \pm 1.3$ \\
$K^\ast_0(1430)^0$ & $0^{+}$ & $\hphantom{0.}1425 \pm 50\hphantom{.}$ & $270 \pm 80\hphantom{0}$ & $\hphantom{.}93 \pm 10\hphantom{.}$ \\
$K^\ast_2(1430)^0$ & $2^{+}$ & $1432.4\pm 1.3$ & $109 \pm 5\hphantom{00}$ & $49.9 \pm 1.2$ \\
$K^\ast(1680)^0$ & $1^{-}$ & $\hphantom{0.}1717 \pm 27\hphantom{.}$ & $322 \pm 110$ & $38.7 \pm 2.5$ \\
$K^\ast_3(1780)^0$ & $3^{-}$ & $\hphantom{0.}1776 \pm 7\hphantom{0.}$ & $159 \pm 21\hphantom{0}$ & $18.8 \pm 1.0$ \\
$K^\ast_4(2045)^0$ & $4^{+}$ & $\hphantom{0.}2045 \pm 9\hphantom{0.}$ & $198 \pm 30\hphantom{0}$ & $9.9 \pm 1.2$ \\
\end{tabular}
\end{center}

\end{frame}

%%%%%%%%%%%%%%%%%%%%%%%%%%%%%%%%%%%%%%%%%%%%%%%%%%%%%%%%%%%%%%%%%
%Slide
%%%%%%%%%%%%%%%%%%%%%%%%%%%%%%%%%%%%%%%%%%%%%%%%%%%%%%%%%%%%%%%%%
\begin{frame}{\BdToKpimm: Angular moments analysis}
 
\begin{itemize}
\item The transversity-basis moments of the first 10 (of 41) orthonormal angular functions
 
 \medskip
 
 \resizebox{0.9\textwidth}{!}{
   \begin{tabular}{c|c|c|c}
     $i$    &   $f_i(\Omega)$             & $\mathrm{\Gamma}^{L, {\rm tr}}_i(\qsq)$ & $\eta^{L\to R}_i$  \\ \hline \hline
     1   &   $P^0_0 Y^0_0$     &  $\left[ \hzsq + \hpasq + \hpesq + \ssq + \dzsq + \dpasq + \dpesq\right]$ & + ($L \to R$)\\ \hline
     2   &   $P^0_1 Y^0_0$     &  $2\left[\frac{2}{\sqrt{5}} \rhzdz + \rshz + \sqrt{\frac{3}{5}}  \rel( H^L_\parallel D^{L\ast}_\parallel + H^L_\perp D^{L\ast}_\perp  )\right]$ & " \\ \hline
     3   &   $P^0_2 Y^0_0$     &  $\frac{\sqrt{5}}{7}$ (\dpasq + \dpesq) - $\frac{1}{\sqrt{5}}$ (\hpasq + \hpesq) + $\frac{2}{\sqrt{5}}$ \hzsq  + $\frac{10}{7\sqrt{5}}$ \dzsq + $2$ \rsdz & " \\  \hline
     4   &   $P^0_3 Y^0_0$     &  $\frac{6}{\sqrt{35}} \left[ - \rel(H^L_\parallel D^{L\ast}_\parallel +  H^L_\perp D^{L\ast}_\perp)  + \sqrt{3} \rhzdz  \right]$ & "\\  \hline
     5   &   $P^0_4 Y^0_0$     &  $\frac{2}{7} \left[ -2 (\dpasq + \dpesq) + 3 \dzsq \right] $ & "\\  \hline
     6   &   $P^0_0 Y^0_2$     &  $\frac{1}{2 \sqrt{5}} \left[ (\dpasq + \dpesq) + (\hpasq + \hpesq) - 2 \ssq - 2 \dzsq - 2 \hzsq \right]$ & " \\  \hline
     7   &   $P^0_1 Y^0_2$     &  $\left[ \frac{\sqrt{3}}{5} \rel(H^L_\parallel D^{L\ast}_\parallel  + H^L_\perp D^{L\ast}_\perp) - \frac{2}{\sqrt{5}} \rel(S^L H^{L\ast}_0)  - \frac{4}{5} \rel(H^L_0 D^{L\ast}_0)\right] $ & "  \\ \hline
     8   &   $P^0_2 Y^0_2$     &  $ \left[ \frac{1}{14} (\dpasq + \dpesq) - \frac{2}{7} \dzsq - \frac{1}{10} (\hpasq + \hpesq) - \frac{2}{5} \hzsq - \frac{2}{\sqrt{5}} \rsdz \right]$ & "  \\  \hline
     9   &   $P^0_3 Y^0_2$     &  $ - \frac{3}{5 \sqrt{7}} \left[ \rel( H^L_\parallel D^{L \ast}_\parallel + H^L_\perp D^{L \ast}_\perp) + 2 \sqrt{3} \rel(H^L_0 D^{L \ast}_0 ) \right] $ & "\\  \hline
     10  &   $P^0_4 Y^0_2$     &  $ -\frac{2}{7 \sqrt{5}}  \left[ \dpasq + \dpesq + 3 \dzsq \right] $ & "  \\
   \end{tabular}
 }
 
\medskip
 
\item The S-, P- and D-wave transversity amplitudes are denoted as $S^{\{L,R\}}$, $H^{\{L,R\}}_{\{0,\parallel,\perp\}}$ and $D^{\{L,R\}}_{\{0,\parallel,\perp\}}$, respectively.
\end{itemize}
\end{frame}

%%%%%%%%%%%%%%%%%%%%%%%%%%%%%%%%%%%%%%%%%%%%%%%%%%%%%%%%%%%%%%%%
%Slide
%%%%%%%%%%%%%%%%%%%%%%%%%%%%%%%%%%%%%%%%%%%%%%%%%%%%%%%%%%%%%%%%%
\begin{frame}{\BdToKpimm: Angular moments analysis}
\begin{itemize}
  \item Distributions of the decays angles within $\pm$50\mevcc of the nominal \Bz mass
  \begin{itemize}
    \item Blue: estimated signal distribution obtained from the angular moments model
    \item Red: projected background from upper mass sideband
  \end{itemize}
\end{itemize}

\medskip

\centering
\includegraphics[width=0.32\textwidth]{figs/kpimm/angular-analysis/costhetal.pdf}
\includegraphics[width=0.32\textwidth]{figs/kpimm/angular-analysis/costhetak.pdf}
\includegraphics[width=0.32\textwidth]{figs/kpimm/angular-analysis/phi.pdf}
\end{frame}

%%%%%%%%%%%%%%%%%%%%%%%%%%%%%%%%%%%%%%%%%%%%%%%%%%%%%%%%%%%%%%%%%
%Slide
%%%%%%%%%%%%%%%%%%%%%%%%%%%%%%%%%%%%%%%%%%%%%%%%%%%%%%%%%%%%%%%%%
\begin{frame}{\BdToKpimm: Systematic uncertainties}

\begin{itemize}
  \item Systematics uncertainties are small ($<30\%$ of statistical uncertainty)
  \item Dominant systematic uncertainty on $\deriv\BF/\deriv\qsq$ due to uncertainty on \mbox{\BF(\BdToJPsiKstP)}
\end{itemize}

\bigskip

\begin{mdframed}[linecolor=barcolor]
\begin{center}
\begin{tabular}{l|cc}
\multicolumn{1}{c|}{Source} & $d\BF/d\qsq \times 10^{-8}~[c^{4}/\gev^{2}]$ & $\overline{\Gamma}_{i}$ \\
\hline
Acceptance stat.\! uncertainty & 0.006--0.030 & 0.003--0.013 \\
Data-simulation differences & 0.001--0.014 & 0.001--0.007 \\
Peaking backgrounds & 0.013--0.026 & 0.001--0.040 \\
\hline
$\BF(\BdToJPsiKstP)$ & 0.033--0.110 & -- \\
\end{tabular}
\end{center}
\end{mdframed}
\end{frame}

%%%%%%%%%%%%%%%%%%%%%%%%%%%%%%%%%%%%%%%%%%%%%%%%%%%%%%%%%%%%%%%%%
%Slide
%%%%%%%%%%%%%%%%%%%%%%%%%%%%%%%%%%%%%%%%%%%%%%%%%%%%%%%%%%%%%%%%%
\begin{frame}{\BdToKpimm: Differential branching fraction}
\begin{center}
\begin{tabular}{lc}
\qsq [$\gevgevcccc$] & $\deriv\BF/\deriv\qsq \times 10^{-8}~[c^{4}/\gev^{2}]$ \\
\hline
$[0.10,0.98]$ & 1.60 $\pm$ 0.28 $\pm$ 0.04 $\pm$ 0.11 \\
$[1.10,2.50]$ & 1.14 $\pm$ 0.19 $\pm$ 0.03 $\pm$ 0.08 \\
$[2.50,4.00]$ & 0.91 $\pm$ 0.16 $\pm$ 0.03 $\pm$ 0.06 \\
$[4.00,6.00]$ & 0.56 $\pm$ 0.12 $\pm$ 0.02 $\pm$ 0.04 \\
$[6.00,8.00]$ & 0.49 $\pm$ 0.11 $\pm$ 0.01 $\pm$ 0.03 \\
\hline
$[1.10,6.00]$ & 0.82 $\pm$ 0.09 $\pm$ 0.02 $\pm$ 0.06 \\
\end{tabular}
\end{center}
\end{frame}

%%%%%%%%%%%%%%%%%%%%%%%%%%%%%%%%%%%%%%%%%%%%%%%%%%%%%%%%%%%%%%%%%
%Slide
%%%%%%%%%%%%%%%%%%%%%%%%%%%%%%%%%%%%%%%%%%%%%%%%%%%%%%%%%%%%%%%%%
\begin{frame}{Angular moments analysis}

\begin{itemize}
\item Normalised moments for $i \in \{2,...,41\}$ defined as
\end{itemize}
\begin{equation*}
\overline{\Gamma}_i(\qsq) = \frac{\Gamma_{i}(\qsq)}{\Gamma_{1}(\qsq)}
\end{equation*}
\begin{itemize}
\item Covariance matrix for $i,j \in \{2,...,41\}$ computed as
\end{itemize}

\begin{equation*}
\overline{C}_{ij} = \left[C_{ij} + \frac{\Gamma_i \Gamma_j}{\Gamma_1^2} C_{11} - \frac{\Gamma_i C_{1j} + \Gamma_j C_{1i}}{\Gamma_1}\right] \frac{1}{\Gamma_1^2}
\end{equation*}

\begin{itemize}
\item[\ding{80}] Can also estimate D-wave fraction
\end{itemize}

\begin{equation*}
F_{\rm D} \equiv  \displaystyle - \frac{7}{18} \left(2\,\overline{\Gamma}_{5} + 5 \sqrt{5}\,\overline{\Gamma}_{10} \right)
\end{equation*}

\end{frame}

%%%%%%%%%%%%%%%%%%%%%%%%%%%%%%%%%%%%%%%%%%%%%%%%%%%%%%%%%%%%%%%%%
%Slide
%%%%%%%%%%%%%%%%%%%%%%%%%%%%%%%%%%%%%%%%%%%%%%%%%%%%%%%%%%%%%%%%%
\begin{frame}{\BdToKpimm: Angular moments analysis}
\begin{columns}
\begin{column}{0.2\textwidth}
\end{column}
\begin{column}{0.6\textwidth}
\centering
\vspace{-1cm}
\resizebox{0.85\textheight}{!}{%
\begin{tabular}{l|c}
$\overline{\Gamma}_{i}$ & Value \\ 
\hline
$\overline{\Gamma}_{2}$ & $-0.42$ $\pm$ 0.13 $\pm$ 0.03 \\ 
$\overline{\Gamma}_{3}$ & $-0.38$ $\pm$ 0.15 $\pm$ 0.01 \\ 
$\overline{\Gamma}_{4}$ & $-0.02$ $\pm$ 0.14 $\pm$ 0.01 \\ 
$\overline{\Gamma}_{5}$ & \hphantom{$-$}0.29 $\pm$ 0.14 $\pm$ 0.02 \\ 
$\overline{\Gamma}_{6}$ & $-0.05$ $\pm$ 0.14 $\pm$ 0.04 \\ 
$\overline{\Gamma}_{7}$ & $-0.06$ $\pm$ 0.15 $\pm$ 0.03 \\ 
$\overline{\Gamma}_{8}$ & \hphantom{$-$}0.04 $\pm$ 0.16 $\pm$ 0.01 \\ 
$\overline{\Gamma}_{9}$ & \hphantom{$-$}0.05 $\pm$ 0.16 $\pm$ 0.02 \\ 
$\overline{\Gamma}_{10}$ & \hphantom{$-$}0.24 $\pm$ 0.17 $\pm$ 0.02 \\ 
$\overline{\Gamma}_{11}$ & \hphantom{$-$}0.06 $\pm$ 0.13 $\pm$ 0.01 \\ 
$\overline{\Gamma}_{12}$ & $-0.01$ $\pm$ 0.13 $\pm$ 0.02 \\ 
$\overline{\Gamma}_{13}$ & $-0.08$ $\pm$ 0.12 $\pm$ 0.01 \\ 
$\overline{\Gamma}_{14}$ & \hphantom{$-$}0.09 $\pm$ 0.13 $\pm$ 0.01 \\ 
$\overline{\Gamma}_{15}$ & \hphantom{$-$}0.11 $\pm$ 0.13 $\pm$ 0.00 \\ 
$\overline{\Gamma}_{16}$ & $-0.12$ $\pm$ 0.13 $\pm$ 0.01 \\ 
$\overline{\Gamma}_{17}$ & $-0.04$ $\pm$ 0.13 $\pm$ 0.01 \\ 
$\overline{\Gamma}_{18}$ & \hphantom{$-$}0.03 $\pm$ 0.14 $\pm$ 0.01 \\ 
$\overline{\Gamma}_{19}$ & \hphantom{$-$}0.11 $\pm$ 0.11 $\pm$ 0.01 \\ 
$\overline{\Gamma}_{20}$ & $-0.00$ $\pm$ 0.11 $\pm$ 0.01 \\ 
$\overline{\Gamma}_{21}$ & \hphantom{$-$}0.03 $\pm$ 0.12 $\pm$ 0.01 \\ 
\end{tabular}
\hspace{1em}
\begin{tabular}{l|c}
$\overline{\Gamma}_{i}$ & Value \\ 
\hline
$\overline{\Gamma}_{22}$ & \hphantom{$-$}0.21 $\pm$ 0.12 $\pm$ 0.01 \\ 
$\overline{\Gamma}_{23}$ & \hphantom{$-$}0.03 $\pm$ 0.12 $\pm$ 0.01 \\ 
$\overline{\Gamma}_{24}$ & $-0.10$ $\pm$ 0.10 $\pm$ 0.01 \\ 
$\overline{\Gamma}_{25}$ & \hphantom{$-$}0.03 $\pm$ 0.10 $\pm$ 0.01 \\ 
$\overline{\Gamma}_{26}$ & \hphantom{$-$}0.08 $\pm$ 0.11 $\pm$ 0.01 \\ 
$\overline{\Gamma}_{27}$ & \hphantom{$-$}0.14 $\pm$ 0.11 $\pm$ 0.01 \\ 
$\overline{\Gamma}_{28}$ & $-0.04$ $\pm$ 0.11 $\pm$ 0.01 \\ 
$\overline{\Gamma}_{29}$ & \hphantom{$-$}0.06 $\pm$ 0.15 $\pm$ 0.04 \\ 
$\overline{\Gamma}_{30}$ & $-0.21$ $\pm$ 0.15 $\pm$ 0.04 \\ 
$\overline{\Gamma}_{31}$ & $-0.07$ $\pm$ 0.16 $\pm$ 0.01 \\ 
$\overline{\Gamma}_{32}$ & $-0.16$ $\pm$ 0.17 $\pm$ 0.02 \\ 
$\overline{\Gamma}_{33}$ & $-0.04$ $\pm$ 0.17 $\pm$ 0.02 \\ 
$\overline{\Gamma}_{34}$ & \hphantom{$-$}0.15 $\pm$ 0.11 $\pm$ 0.01 \\ 
$\overline{\Gamma}_{35}$ & $-0.13$ $\pm$ 0.11 $\pm$ 0.01 \\ 
$\overline{\Gamma}_{36}$ & \hphantom{$-$}0.05 $\pm$ 0.11 $\pm$ 0.01 \\ 
$\overline{\Gamma}_{37}$ & \hphantom{$-$}0.05 $\pm$ 0.11 $\pm$ 0.01 \\ 
$\overline{\Gamma}_{38}$ & \hphantom{$-$}0.06 $\pm$ 0.11 $\pm$ 0.00 \\ 
$\overline{\Gamma}_{39}$ & $-0.08$ $\pm$ 0.11 $\pm$ 0.00 \\ 
$\overline{\Gamma}_{40}$ & \hphantom{$-$}0.15 $\pm$ 0.11 $\pm$ 0.01 \\ 
$\overline{\Gamma}_{41}$ & \hphantom{$-$}0.12 $\pm$ 0.11 $\pm$ 0.01 \\ 
\end{tabular}
}
\end{column}
\begin{column}{0.2\textwidth}
\end{column}
\end{columns}
\end{frame}

%%%%%%%%%%%%%%%%%%%%%%%%%%%%%%%%%%%%%%%%%%%%%%%%%%%%%%%%%%%%%%%%%%%%%%%%%%%%%%%%%%%%%%%
%%%%%%%%%%%%%%%%%%%%%%%%%%%%%%%%%%%%%%%%%%%%%%%%%%%%%%%%%%%%%%%%%%%%%%%%%%%%%%%%%%%%%%%
%           END OF DOCUMENT
%%%%%%%%%%%%%%%%%%%%%%%%%%%%%%%%%%%%%%%%%%%%%%%%%%%%%%%%%%%%%%%%%%%%%%%%%%%%%%%%%%%%%%%
%%%%%%%%%%%%%%%%%%%%%%%%%%%%%%%%%%%%%%%%%%%%%%%%%%%%%%%%%%%%%%%%%%%%%%%%%%%%%%%%%%%%%%%
\end{document}


